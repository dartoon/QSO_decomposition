\documentclass{natureprintstyle}
%\documentclass{nature}
\bibliographystyle{naturemag}
\usepackage{epsfig,caption}
\usepackage{color}
\usepackage{bm}
\usepackage{graphicx}
\usepackage{longtable}
\usepackage{amssymb}
\usepackage{rotating}
\usepackage{latexsym}
\usepackage{hyperref}
\usepackage{float}

%\usepackage[switch]{lineno}
%\linenumbers

%%%%%%%%%%%%%%%%%%%%%%%%%%%%%%%%%%%%%%%%%%%%%%%%%%%%%%%%%%%%%%%%%%%
% comment for selecting the best figures to go on web for:
%@arxiver{jvds_sami_vsigma_ellip_LOESS_Age_paper.pdf,jvds_sami_vsigma_ellip_LOESS_Age_transform_paper_resubmit.pdf} 
%%%%%%%%%%%%%%%%%%%%%%%%%%%%%%%%%%%%%%%%%%%%%%%%%%%%%%%%%%%%%%%%%%%

%journal commands
\newcommand{\apj}{Astrophys. J.}
\newcommand{\spie}{Proc. SPIE}
\newcommand{\pasp}{Publ. Astron. Soc. Pac.}
\newcommand{\apjs}{Astrophys. J. Supp.}
\newcommand{\araa}{Annu. Rev. Astron. Astrophys.}
\newcommand{\mnras}{Mon. Not. R. Astron. Soc.}
\newcommand{\apjl}{Astrophys. J. Let.}
\newcommand{\aap}{Astron. Astrophys.}
\newcommand{\aj}{Astron. J.}
\newcommand{\nat}{Nature}
\newcommand{\na}{New Astron. Rev.}
\newcommand{\aaps}{A\&AS}
\newcommand{\procspie}{Proc. SPIE}

%%%%%%%%%%%%%%%%%%%%%%%%%%%%%%%%%%%%%%%%%%%%%%%%%%%%%%%%%%%%%%%%%%%
% my commands

\newcommand{\lcdm}{$\Lambda$CDM}
\newcommand{\hst}{{\it HST}}
\newcommand{\efr}{$R_{\mathrm{eff}}$}
\newcommand{\galfit}{{\sc Galfit}}
\newcommand{\mbh}{$\mathcal M_{\rm BH}$}
\newcommand{\lhost}{$L_{\rm host}$}
\newcommand{\mr}{$Mag_{\rm ~R}$}
\newcommand{\jcap}{Journal of Cosmology and Astroparticle Physics}
\newcommand{\halpha}{${\it H}\alpha$}
\newcommand{\hbeta}{${\it H}\beta$}
\newcommand{\sersic}{S\'ersic}
\newcommand{\lenstronomy}{{\sc Lenstronomy}}
\newcommand{\reff}{{$R_{\mathrm{eff}}$}}
%\newcommand{\kms}{km~s$^{\rm -1}$}
\newcommand{\kms}{\ifmmode{\,\rm{km}\, \rm{s}^{-1}}\else{$\,$km$\,$s$^{-1}$}\fi}
\newcommand{\sigstar}{{$\sigma_*$}}
\newcommand{\mstar}{{$M_*$}}
\newcommand{\Mgii}{Mg$_{\rm II}$}
\newcommand{\Civ}{C$_{\rm IV}$}
\newcommand{\farcs}{\mbox{\ensuremath{.\!\!^{\prime\prime}}}}% fractional arcsecond symbol: 0.''0
\newcommand{\sam}{\texttt{SAM}}
\newcommand{\mbii}{\texttt{MBII}}

%%%%%%%%%%%%%%%%%%%%%%%%%%%%%%%%%%%%%%%%%%%%%%%%%%%%%%%%%%%%%%%%%%%

\newcommand{\ding}[1]{\textcolor{red}{[{\bf Xuheng}: #1]}} 

\title{
%From predictions to observation: scaling relations between supermassive black holes and their host galaxies at $1< z<2$
A successful observational test of black hole and galaxy co-evolution models since $z\sim1.7$
%The first comparison between the observation and simulation of the scaling relations between supermassive black holes and their host galaxies at $1.2< z<1.7$
}
\author{Xuheng Ding$^{1,2}$, 
Tommaso Treu$^{1}$, 
John Silverman$^{3, 4}$,
Aklant K. Bhowmick$^{5}$,
N. Menci$^{6}$,
Tiziana Di Matteo$^{5}$,
et. al.
}

\begin{document}

\maketitle

\let\thefootnote\relax\footnote{
\begin{affiliations}
\item {Department of Physics and Astronomy, University of California, Los Angeles, CA, 90095-1547, USA} 
\item {School of Physics and Technology, Wuhan University, Wuhan 430072, China}
\item {Kavli Institute for the Physics and Mathematics of the Universe (WPI), The University of Tokyo, Kashiwa, Chiba 277-8583, Japan}
\item {Department of Astronomy, School of Science, The University of Tokyo, 7-3-1 Hongo, Bunkyo, Tokyo 113-0033, Japan}
\item{McWilliams Center for Cosmology, Dept. of Physics, Carnegie Mellon University, Pittsburgh PA 15213, USA}
\item{INAF Osservatorio Astronomico di Roma, via Frascati 33, I-00078 Monteporzio, Italy}
\end{affiliations}
}

\begin{abstract}
Supermassive black holes (BH) are known to reside at the center of massive galaxies. In the local Universe, BH masses (\mbh) are tightly correlated to the properties of their host galaxies, including absolute magnitude ($Mag$) and stellar mass (\mstar), known as scaling relations (i.e., \mbh-\lhost, \mbh-\mstar). The origin of this correlation is still not fully understood, and one promising way to reveal it is to use simulation. During the past few years, simulation have successfully reproduced the observational relations over the last eight billion years ($z\lesssim1$). However, due to the limitation of high-$z$ measurements, the simulations haven't yet make the comparison to the observations at earlier cosmic epochs ($z>1$), where the predictions are more sensitive to the initial assumptions and any inconsistency would help to pinpoint missing physics.
For this purpose, we adopted a unique sample of 32 broad-line Active Galactic Nuclei (AGN) at $1.2 < z < 1.7$ ($\sim 8.5-10$ billion years ago) with the broad \halpha\ emission line as detected by Subaru/FMOS, providing an accurate assessment of \mbh, while multi-band \hst\ imaging is taken to derive the host galaxies properties (e.g., \mstar).
We directly compare the measurements to two independent state-of-the-art simulations: \texttt{MassiveBlackII} (MBII) and semi-analytic model (SAM).  Taking the selection biases into account, %by adopting the same selection function to choose the simulating sample,
we find that both \mbii\ and \sam\ agree well with the data, in terms of the central distribution. However, when considering the measurement uncertainty and comparing the scatters, the observations are significantly more consistent with the \mbii\ sample ($\sim0.3$~dex), than the \sam\ ($\sim0.7$~dex). {\color{red} In this respect, our observational constraints are able to distinguish the recipe adopted between the two projects. For example, in the \mbii, it is the AGN feedback that drives the scaling relations, while in the \sam\ it is galaxy merger driven.} Moreover, the intrinsic scatter of our high-$z$ sample is comparable to the local sample thus counter to the expectation from a scenario based on the central limit theorem (a purely stochastic process). To some extent, our result attests that it is the AGN feedback that establishs a causal link between SMBH and its host galaxy to induce an ordering so that the more massive black holes are tightly linked to the more massive galaxies.

\end{abstract}

%\section{Introduction}
The discovery of the correlations between the masses (\mbh) of supermassive black holes (SMBH) and the properties of their host galaxies, such as absolute magnitude ($Mag$) and stellar mass (\mstar), indicates that the growth of SMBH could be connected to the formation of their host galaxies~\cite{Mag++98, F+M00, M+H03, H+R04, Gul++09}. So far, the origin that drives this connection is still unknown, due to the daunting range of scales between the dynamical sphere ($\sim$pc) of the SMBH and their host galaxy ($\sim$10 kpc). A possible physical link may be feedback from an Active Galactic Nucleus (AGN) phase, assuming that a small fraction of the AGN energy is injected into their surrounding gas that regulates the growth of the SMBH and its host galaxy. In this scenario, AGN activity heats and unbinds a significant fraction of the gas and inhibits star formation. However, there are other models that have been proposed to explain the connection between SMBH and their hosts such as through an indirect manner where AGN accretion and star formation are fed through a common gas supply~\cite{Cen2015, Menci2016}. Furthermore, it has been proposed that the statistical convergence from galaxy assembly alone (i.e., dry mergers) may reproduce the observed correlations without any direct physical mechanisms~\cite{Peng2007, Jahnke2011, Hirschmann2010}. From the central limit theorem, a stochastic cloud at high-$z$ (higher dispersion) would end up with scaling relations as observed today with lower dispersion.

To understand this connection, recent works~\cite{Park15, Tre++07, Bennert11, Woo++08} have measured the scaling relations out to intermediate redshift {($0.3\lesssim z \lesssim1$)} by studying AGN and their host galaxies with {\it Hubble Space Telescope} ({\it HST}) and found an `observed' evolution in which the growth of SMBHs predates their host galaxies. However, the observational evidence of such evolution at high-$z$ is under debated since it depends on our understanding of systematic uncertainties and selection effects~\cite{Lauer2007}. It has been realized that an improper consideration of the uncertainties and selection effects will lead to an {\it apparent} evolution and result in an overestimate of the evolution\cite{Volonteri2011}.

%Why comparison between data and model.
%Summary of recently works.
%The reason to extend to high redshift.
Simulations are effective at helping to understand this connection and rule out theories/assumptions that could not be definitively verified by observations alone. In particular, simulations can be used to quantify the impact of systematic uncertainties and selection biases with observational analysis. Recent studies, based on several simulation projects, have made predictions that are in good agreement with the observational data at intermediate redshifts ($z\lesssim1$). For example, the state-of-the-art cosmological hydrodynamical simulation of structure formation (\texttt{MassiveBlackII}) has been utilized to compare the predicted scaling relations to \hst\ observation at $0.3<z<1$ that show a positive evolution where the SMBH growth predates that of its the host galaxy~\cite{DeG++15}. Several other works have investigated scaling relations using large-volume simulations, resulting in good agreement with the local relation and some redshift evolution, including the Magneticum Pathfinder SPH Simulations~\cite{Steinborn2015}, the Evolution and Assembly of GaLaxies and their Environments (EAGLE) suite of SPH simulations~\cite{Schaye2015}, and Illustris moving mesh simulation\cite{Sijacki2015, Vogelsberger2014, Li2019}. Besides hydrodynamic simulations, semi-analytic models~\cite{Menci2014, Menci2016} have also made remarkable progress and recovered the local scaling relations~\cite{Kormendy13}. However, to date, the comparisons to the observed scaling relations by high-resolution \hst\ imaging sample are still limited within $z<1$ due to the limitation of the observations.

% The work that we want to do
	%Sample selection
	%The advantage of our measurements
	%Compare to which simulation? (some introduction)
	%Any expectation?
	%mention the scatter.
To extend the range, we have been studying a sample of 32 \hst-observed AGN systems at redshift range $1.2<z<1.7$ from three X-ray coverage fields (D19~\cite{Ding2019} hereafter), including COSMOS~\cite{Civano2016}, (E)-CDFS-S~\cite{Lehmer2005, Xue2011}, and SXDS~\cite{Ueda2008}. The redshift range of our targets is chosen to be high enough ($z>1$) to detect the existence of the evolution offsets and make the comparison to the models, while low enough ($z<2$) to limit the effect of surface brightness dimming. 
We selected our AGN sample in a well-defined window based on the \mbh\ and Eddington ratio, as shown in Figure~1 in D19. As shown in that Figure, the \mbh\ are well below the knee of the BH mass distribution to avoid strong selection bias. In addition, the Eddington ratios are mostly above $0.1$ to ensure the homogeneity.

We measure reliable \mbh\ and host properties to overcome the systematic. Specifically, the \mbh\ of our sample are estimated using the published near-infrared spectroscopic observations of the broad \halpha\ emission lines, which removes the systematic with \Mgii\ or \Civ. For the inference of properties of the host galaxies, the X-ray selected sample has lower nuclear-to-host ratios at IR band, which facilitates the inference of the host light. We employ the \hst/WFC3 to obtain high-resolution (0\farcs0642 per pixel) imaging data and adopt the state-of-the-art technique to perform the 2-D flux profile decomposition to infer the light of the host galaxy. Moreover, 21/23 systems have \hst/ACS imaging data, which enable us to assess the colors of the host, hence the rest frame R-band magnitude (\mr) and stellar mass (\mstar) (see Methods and the descriptions in D19). 
%We furthermore combining them with the ground-based photometry to carry out the SED fitting, and thus derive the reliable host rest-frame R band luminosity and stellar mass.

%\begin{figure}[t]
%\includegraphics[width=0.9\linewidth]{AGN_selection.pdf}
%\caption{The selection for the 32 AGN sample. The top and right panels are the BH mass function and Eddington ratio function. \ding{More descriptions are needed if Figure1 is finally decided to show.}
%}
%\label{fig:AGN_select}
%\end{figure}

To make the comparison, we adopt two independent simulation projects, including \texttt{MassiveBlackII} (MBII)~\cite{Khandai2015} and semi-analytic models (SAM)~\cite{Menci2014}. These two simulations are based on independent model strategies, i.e., hydrodynamic simulation for \mbii\ and semi-analytic model for \sam, respectively. The \mbii\ simulation is the highest resolution at the size of a comoving volume $V_{\rm box} = (100~{\rm Mpc}~h^{-1})$, including a self-consistent model for star formation, black hole accretion, and associated feedback. The large simulation volume enables the simulating objects to evolve independently; the high enough mass and spatial resolution meet the requirements for the object details. On the theoretical side, aimed, high-resolution N-body simulations can study specific galaxy systems. However, understanding the mechanism of the scaling relation requires an analytical description of such processes to be implemented into existing semi-analytic models, such as \sam. In previous works, \mbii~\cite{Huang2018, DeG++15, Khandai2015,Bhowmick2019} and \sam~\cite{Menci2014, Menci2016} have made high successful predictions, and we refer the interested readers to the Method section for more details.

%\section*{Results}
%Selection effect into account
Based on \mbii\ project, we collect a sample of simulated AGNs at $z=1.5$ and compare their predicted scaling relations to the observed ones. We take the measurement uncertainty and the selection effect into account to ensure a fair comparison,  with the following process. First, we inject the random noise to the simulated sample to mimic the scattering effect caused by the uncertainty. We assume uncertainties are comparable to the realistic levels, i.e., $\Delta$\mbh =0.4~dex, $\Delta$\mr=0.3~dex, $\Delta$\mstar=0.17~dex, and $\Delta L_{\rm bol}$=0.03~dex, respectively. We then select the sample that falls into the same targeting window to mimic the realistic sample selection process, as shown in Figure~\ref{fig:selectfunc}. Considering that the mass relations between SMBH and its host are more substaintial, we focus on the predicted \mbh-\mstar relations and compared to the observational constraints, as shown in Figure~\ref{fig:MM_comp} (left panel). From a quick glimpse, one can see that the samples are overlapped with each other, showing a sign of agreement. To quantify the agreement level, we perform the linear regression to fit the simulated relations and estimate the best-fit inference and 1$-\sigma$ confidence interval. To make direct comparison, we fix the same slope value and obtain the best-fit intercept value for the observation. We find that, for the \mbh-\mstar\ relations, there is a notable mismatch ($\sim0.4$~dex), however within 1$-\sigma$ level. Furthermore, we compare the scatter of the sample by calculating the standard derivation of the residual in the linear regression. We find that the simulations and the observations have the same scatter level $0.3$~dex. Considering that the simulated samples have the same uncertainty and the selection effect, we expect the intrinsic scatter of the scaling relations are similar, which is $0.25$~dex (see METHODS).

\begin{figure}[t]
\includegraphics[width=1.2\linewidth]{MBII_selectfunc.pdf}
\caption{The same selection window as adopted for selecting the \mbii\ simulated sample. The brown background clouds shows the overall \mbii\ sample at $z=1.5$. We add the random uncertainty to the simulation and select the ones fall into the targeting region (i.e. blue colored). The orange dots are the \hst\ observed sample.}
\label{fig:selectfunc}
\end{figure}
%comparison to simulation (flux ratio also)

\begin{figure*}[t]%[!b]
\begin{tabular}{c c}
\includegraphics[trim = 0mm 0mm 65mm 0mm, clip, width=0.47\linewidth]{MBII_MM.pdf} &
\includegraphics[trim = 0mm 0mm 65mm 0mm, clip, width=0.47\linewidth]{SAM_MM_consider_nois.pdf} \\
\end{tabular}
\caption{In the left panel, we present the comparison of the  \mbh-\mstar\ correlation between the observation (orange dots) and the \mbii\ predicted samples (blue dots). The predicted sample is treated to have the same uncertainty and selection effect as the observational ones. The blue line is the best-fit result for the \mbii\ sample, with the colored region indicating the $1-\sigma$ confidence interval. We use the same slope value to fit for the observed sample, and the orange line shows the best-fit result. The brown grids in the background are the overall sample that predicted by the \mbii\ simulation. We present the comparison with the \sam\ sample (green color) in the right panel.
}
\label{fig:MM_comp}
\end{figure*}

We also make comparisons of our observed scaling relations to the predictions by the \sam\ model. Rather than N-body simulation which produces individual simulating objects, the \sam\ model uses the density clouds to describe the possibility of the samples. 
%We consider the sample at $z=1.5$ epoch for two sample sets, including the overall sample and the selected sample, as shown in Figure~\ref{fig:SAM_comp}. We find that considering the selecting effect, the direct prediction by \sam\ is well matched to the observed relations.
%We do not estimate the scatter for \sam\ model, but visually see that the observed relations are mostly located on top of the hot cloud, showing that the observed scatter should not be larger than the model one. \ding{This is not right yet. But if consider the scatter in the same way, the \sam\ have large scatter?}
%Yet this is not a proper comparison since the direct sample from \sam\ does not include the scattering effect by the measurement uncertainty.
To make the direct comparison, we first randomly generate an overall \sam\ sample based on the possibility clouds at $z=1.5$ epoch. Then, same as the process in \mbii\ analysis, we inject the random uncertainty to the sample to take the uncertainty effect into account and select the sample the selection window. The resulting comparison of the  \mbh-\mstar\ relation are shown in Figure~\ref{fig:MM_comp} (right panel). We find that the best-fit result by the \sam\ model is well matched to the observation. However, the scatter of the \sam\ model are overwhelmingly larger ($0.7$~dex) than the observation. This larger scatter is a combination result by the \sam\ distribution, uncertainty, and selection effect.  


%\begin{figure*}[t]%[!b]
%\begin{tabular}{c c}
%\includegraphics[trim = 0mm 0mm 24mm 0mm, clip, width=0.45\linewidth]{SAM_ML.pdf} &
%\includegraphics[trim = 30.5mm 0mm 0mm 0mm, clip, width=0.45\linewidth]{SAM_MMstar.pdf} \\
%\end{tabular}
%\caption{Similar to the Figure~\ref{fig:MM_comp}, but for a direct comparison to the observation using SAM's model. The blue background contours show the possibility distribution for the overall sample, and the red contours indicate the distributions after considering the selecting effect. We presented our observational data as yellow dots. Note that this comparison has not taken the measurement uncertainties into account yet. 
%}
%\label{fig:SAM_comp}
%\end{figure*}

 \ding{We can put the following paragraph into Method or even remove it.} To test if any unexpected selection effects exist, we compare the distribution of the host-total flux ratio among these three samples. For the observed sample, we calculate the flux ratio at the imaging band, i.e., \hst/WFC3. For the simulated sample, we consider the AGN bolometric correction~\cite{Elvis1994} to estimate the AGN light flux at WFC3/F125W band. We compare their host-total flux histogram distribution in Figure~\ref{fig:comp_hist} and see that the three samples are well matched each other. The median values for the flux ratio distribution of the observed, \mbii, \sam\ sample are $37.3\%$, $32.3\%$, and $42.8\%$, respectively. We perform the Kolmogorov-Smirnov (KS) test the inferred p-values are 0.34 (for observed -- \mbii) and 0.14 (for observed -- \sam), respectively. This result indicates the three samples have similar host-to-total light distribution.

\begin{figure}[t]
\includegraphics[width=0.9\linewidth]{comp_host_ratio.pdf}
\caption{The comparison histogram of the host ratios among three referred samples, median value indicated. The same selection effect are considered for the simulation sample.
}
\label{fig:comp_hist}
\end{figure}

{\bf In terms of their central distribution, the \mbii\ simulation and \sam\ model both demonstrate an excellent agreement to the observational constraints. When considering the scattering effect by the measurement uncertainty, the scatter in \mbii\ simulation is consistent to the observational one ($\sim0.3$~dex); however, \sam\ sample has much larger scatter ($\sim0.7$~dex). Given that the two models adopt different assumptions, recipes, and initial conditions, our result shows that the mechanism by \mbii\ is more favorable, in which it is the AGN feedback that drives the scaling relation \ding{To Aklant and Tiziana, do you agree with this statement?}. Though \sam\ model also consider the AGN feedback, the feeding process for the SMBH accretion is still galaxies merger driven with more stochasticities in the scaling relations, which may leads to the larger scatter. In particular, in the \sam\ model the encounters are assumed to trigger the feedback, and the fraction of gas that feeds the SMBH is related to the parameters of the 
encounter. This introduces additional scatter since it depends on the properties of both interacting galaxies (see Method). As a result, the \sam\ cloud extends to the high \mbh\ with low stellar mass, which may not exist. \ding{To Nicola: do you want to include your new 2D model in the paper?} The consistency between \mbii\ and observations indicates that there has to be a causal link (i.e., AGN feedback) between SMBH and its host galaxy to generate an ordering during their evolution; as a result, more massive BHs are linked to more massive galaxies {\it tightly}.
}

Without any physical mechanisms, the central limit theorem~\cite{Peng2007, Jahnke2011, Hirschmann2010} assumes that the scaling relations are caused by the random merger from a stochastic cloud at earlier universe. In this scenario, the scatter of the scaling relations has to be larger with the increasing of redshift. However, the observed scatter of our high-$z$ sample do not present this feature. In fact, our inferred intrinsic scatter of the observed sample is $\sim0.25$~dex, which is no more than the typical scatter at local relations in the literature ($\sim0.35$~dex)~\cite{Kormendy13, Gul++09}. Of course, the intrinsic scatter of our high-$z$ sample could be inaccurate, since we use the \mbii\ overall sample as a proxy to estimate the level for the real data.
Also, the observed \mbh\ are estimated using the robust \halpha\ line, which could have lower uncertainties level than expected (i.e., $\Delta$\mbh$<0.4$~dex), resulting in an overestimating of the error-budget and thus underestimating of intrinsic scatter. Still, the evidence of the low scatter in our measurements is likely to be true, which is against the central limit theorem and suggests that the physical mechanisms between the SMBH and its host galaxy are likely to exist. 

Extending the redshift range of this study would be very beneficial to this study, giving that the scaling relation in the simulations shows different evolution at the different stage of the universe. For higher redshift, the {\it James Webb Space Telescope} may provide high-quality imaging data of AGNs at redshift up to $z\sim7$. In the low redshift Universe, wide-area surveys with Subaru/HSC, LSST, and WFIRST offer much promise to build samples for studying these mass ratios and dependencies on other factors (e.g., environment).

%\textcolor{blue}{The details of the results could be presented.\\
%0. Compare the color of the sample?\\
%1. 2D KS test? \\
%2. More insightful inference from the comparing?
%}

%Discuss the result, What does this mean?

% \textcolor{blue}{If the final results show such feature:}
%The result also shows that the \mbh-\mstar\ relation has less scatter than the \mbh-\lhost\ relation, suggesting that the BH relation with \mstar\ are more fundamental than \lhost.
%This successful experience leads one conjecture that the other prediction by the simulations could likely to be right. For example, regarding the stellar components of galaxy which is the origin that related to the growth of BH, it is likely that the bulge masses are tightly related.

%\section*{Discussion}
%Some discussions should be placed in this section.
%On the observation side:\\
%0. The inferred host flux ratio by SED decomposition is consistent to the 2-D image decomposition, indicating the fidelity of these approaches?\\
%1. \hst\ seems to reach its higher limit. In the future, the JWST is very promising to realize the evolution scenario at even higher redshift.\\
%On the simulation side:
%\\0. Introduce some the other 
%\\1. How do we discuss the role of AGN feedback?.

\begin{center}
{\bf \Large \uppercase{Methods} }
\end{center}

%\textbf{Observation data.} Our 32 new AGN systems are selected from four X-ray coverage fields including COSMOS (Civano et al. 2016), (E)- CDFS-S (Lehmer et al. 2005; Xue et al. 2011), and SXDS (Ueda et al. 2008) at redshift range $1.2<z<1.7$. The X-ray selected sample have low nuclear-to-host ratios, which facilitates the extraction of the host properties. We adopt the \hst/WFC3 infrared channel to derive the high spatial resolution imaging data, to carry out the decomposition of the AGN-host using two-dimensional flux distribution. The details of the \hst\ observation and the study are presented in the companion paper. Moreover, 21/32 systems have \hst/ACS band, together with some other ground-based observations, which would provide the host information in the other bands. In the next section, we infer the reliable K-correction for the rest-frame R band luminosity and the SED to infer the stellar mass. The \mbh\ of our sample have been estimated by \halpha\ and \hbeta\ in the FMOS survey. Comparing to the \Mgii\ and \Civ, the \mbh\ by broad Balmer lines are more trustworthy. The estimated value of the \mbh\ are listed in the companion paper.
%\textcolor{blue}{Do we need to list the \mbh\ and host properties in a table in this paper?}

\textbf{Imaging data, decomposition and inference of host properties.} 
%1. Imaging data. 2. Inferring host light. 3. Color inference. Rest frame R band and stellar mass.
The detailed description for the analysis is in the companion paper (referring as D19). Here we only present a quick overview of this study. We adopted the \hst/WFC3 infrared channel and selected to use the filters F125W $(1.2<z<1.44)$ and F140W $(1.44<z<1.7)$ (according to the redshift of the targets) to observe the imaging data for the 32 AGN systems through the \hst\ program GO-15115 (PI: John Silverman). We obtain six dither exposures with total exposure time $\sim2394s$ and using  {\sc astrodrizzle} software package to co-add the final image with pixel scale as 0\farcs0642. To mitigate the contamination by the background light from both the sky and the detector, we adopt the {\sc photuils} tools to estimate and remove them accurately.

To address the bias which might be raised by the unknown PSF, we manually collect the isolated-unsaturated PSF-stars from the 32 observed fields, to assemble a PSF library for the fitting. To decompose each AGN image, we assume the unresolved active nuclei as the scaled point source and the host galaxy as the \sersic\ profile. We adopt imaging modeling tool \lenstronomy\cite{lenstronomy} to simultaneously fit their 2-D flux distribution, taking each PSF one by one from the library. Based on the reduced $\chi^2$, we are capable of evaluating the performance of each PSF. We adopt the result from the top-ranked-eight PSFs and using the weighting process to obtain the host property, including flux, \reff, \sersic\ index, using a weighted arithmetic mean.

21/32 images in the COSMOS survey also have imaging data in the ACS/F814W with drizzled pixel scale as 0\farcs03, providing us the multi-band information. We decompose the AGN image in the ACS band to obtain the host flux and compare to the WFC3's result to infer the host color. {\bf We find that the $1$~Gyr and $0.625$~Gyr stellar templates could well match the sample color at $z<1.44$ and $z>1.44$ respectively, see Figure~5 in D19, from which we estimate the rest-frame R band luminosity and stellar mass.} A Salpeter initial mass function is employed consistently to the observed and simulated sample.

%\textbf{SED fitting and host properties}
%We perform the SED fitting to derive the robust physical properties of the host galaxy, including the rest-frame R band luminosity and stellar mass. 
%21/32 AGN systems have rest-frame UV imaging data by \hst-ACS/F814W\cite{Scoville2007}. We perform the same approach to decompose the photometry of the host light at UV. Since the host inference by IR band is superior to the one by UV band, while fitting UV image, we fix the \reff\ and \sersic\ index as the value inferred by \hst/WFC3 to derive the host flux.
%We furthermore combine the \hst\ inference to the ground-based AGN photometry to carry out the SED fitting. \textcolor{blue}{Details and the figures need to be presented.}
%Adopting the best fit SED stellar template to the \hst/WFC3 flux, we derive the \lhost${,_R}$ and \mstar.

%\textbf{Simulations.} 
%Describe the approaches used in the simulation.

\textbf{The \texttt{MassiveBlackII} (\mbii) simulation}.  
\mbii\ is a high-resolution cosmological hydrodynamic simulation which ran from $z~=~159$ to $z~=~0.06$ using the Smooth Particle Hydrodynamics~(SPH) code \texttt{P-GADGET}, which is an upgraded version of~\texttt{GADGET-2}~\cite{2005MNRAS.364.1105S}. It has a box size of $100~\mathrm{cMpc/h}$ and $2\times1792^3$ particles. The resolution elements for dark matter and gas have masses of $1.1\times 10^7~M_{\odot}/h$ and $2.2\times 10^6~M_{\odot}/h$, respectively. The base cosmology corresponds to the results of WMAP7 \cite{2011ApJS..192...18K}, i.e., $\Omega_0=0.275$, $\Omega_l=0.725$, $\Omega_b=0.046$, $\sigma_8=0.816$, $h = 0.701$, $n_s=0.968$.  The simulation includes a full modeling of gravity + gas hydrodynamics, as well as a wide range of subgrid recipes for the modeling of star formation \cite{2003MNRAS.339..289S}, black hole growth and feedback processes. Haloes were identified using a Friends-of-Friends (FOF) group finder \cite{1985ApJ...292..371D}. Within these haloes, self-bound substructures/subhaloes were identified using \texttt{SUBFIND} \cite{2005MNRAS.364.1105S}. Galaxies are identified to the stellar matter components of the subhaloes.

For the modeling of black hole growth and associated feedback, we adopt the prescription in the reference \cite{2005Natur.433..604D, 2005MNRAS.361..776S}. In particular, seed black holes of mass $5\times 10^{5}~M_{\odot}/h$ are inserted into haloes of mass $\gtrsim 5\times 10^{10}~M_{\odot}/h$~(if they do not already contain a black hole). Once seeded, black hole growth occurs via gas accretion at a rate given by $\dot{M}_{bh}={4\pi G^2 M_{bh}^2 \rho}/{(c_s^2+v_{bh}^2)^{3/2}}$ where $\rho$ and $c_s$ are the density and sound speed of the ISM gas~(cold phase); $v_{bh}$ is the relative velocity between the black hole and the gas in its vicinity. $10\%$~(radiative efficiency) of the accreted gas is released as radiation. The accretion rate is allowed to be mildly super-Eddington, i.e., limited to two times the Eddington accretion rate. A fraction~($5\%$) of the radiated energy couples to the surrounding gas as black hole~(or AGN) feedback \cite{2005Natur.433..604D}. For the modeling of black hole mergers, two black holes are considered to be merged if their separation distance is within the spatial resolution of the simulation~(the SPH smoothing length), and their relative speeds are lower than the local sound speed of the medium.

For the galaxy photometry, the spectral energy distributions~(SEDs) of the host galaxies were first obtained by summing up the contributions from the individual star particles. The stellar SEDs were modelled using the \texttt{PEGASE-2}~\cite{1999astro.ph.12179F} stellar population synthesis code with a Salpeter IMF. The galaxy SEDs are finally convolved with the desired filter function to obtain the broad band photometry~(SDSS $r$ band magnitude). 

For more details regarding the \mbii\ simulation, we refer the reader to the reference~\cite{2015MNRAS.450.1349K}.

\textbf{The semi-analytic model (\sam)}. The \sam\ is based on the description in Menci et al. (2016)~\cite{Menci2016}, to which we refer for details; here we recall its key points. 
The merging trees of dark matter haloes are generated through a Monte Carlo procedure adopting the merging rates given by the Extended Press \& Schechter formalism~\cite{Lacey1993} assuming a Cold Dark Matter power spectrum of perturbations. For each dark matter halo included into a larger halo we computed the dynamical friction process, to  determine whether it will survive as a satellite, or sink to the centre to increase the mass of the central dominant galaxy; binary interactions (fly-by and merging) among satellite sub-halos are also described by the model. In each halo, we compute the amount of gas which cools due to atomic processes. The cooled gas  settles into a rotationally supported disc as the descriptions in the reference~\cite{Mo1998}. The gas is converted into stars through three different channels: quiescent star formation, gradually converting the gas into stars over long timescales $\sim 1$~Gyr; starbursts following galaxy interactions, occurring on timescales $\lesssim 100$~Myr, associated to BH feeding; internal disc instabilities triggering loss of angular momentum resulting into gas inflows toward the centre, therefore feeding star formation and BH accretion. The energy released by the Supernovae associated to the total star formation returns a fraction of the disc gas into the hot phase (stellar feedback). 

The semi-analytic model includes BH growth from primordial seeds. These are assumed to originate from PopIII stars with a mass $M_{seed}=100\,M_{\odot}$~\cite{Madau2001}, and to be initially present in all galaxy progenitors. We consider two BH feeding modes: accretion triggered by galaxy interactions and internal disc instabilities. These are described in detail in our previous work~\cite{Menci2016}, and briefly recalled below.\newline
1) BH accretion triggered by interactions. The interaction rate $\tau_r^{-1}=n_T\,\Sigma (r_t,v_c,V)\,V_{rel} (V)$ for galaxies with relative velocity $V_{rel}$ and number density $n_T$ in a common DM halo determines the probability for encounters, 
either fly-by or  merging, through the corresponding cross sections $\Sigma$ given in Menci et al. (2014)~\cite{Menci2014}. The fraction of
gas destabilized in each interaction corresponds to the loss $\Delta j$ of orbital angular momentum $j$, and depends on the mass ratio of the merging partners $M'/M$ and on the impact factor $b$. \newline
2) BH accretion induced by disc instabilities. We assume these to arise  in  galaxies with disc mass exceeding~\cite{Efstathiou1982} $M_{crit} =  {v_{max}^2 R_{d}/ G \epsilon}$ with $\epsilon=0.75$, where $v_{max}$ is the maximum circular velocity associated to each halo \cite{Mo1998}. 
Such a criterion strongly suppresses the probability for disc instabilities to occur not only in massive, gas-poor galaxies, but also in 
dwarf galaxies characterized by small values of the gas-to-DM mass ratios.
The instabilities induce loss of angular momentum resulting into strong inflows that we compute following the 
description in Hopkins et al. (2011)~\cite{Hopkins2011}, recast and extended as in Menci et al. (2014)~\cite{Menci2014}. 

Finally, the \sam\ model includes a detailed treatment of AGN feedback, presented and discussed in Menci et al. (2008)~\cite{Menci2008}.
This is assumed to stem from the fast winds with velocity up to
$10^{-1}c$ observed in the central regions of AGNs~\cite{Chartas2002, Pounds2003}.  
These supersonic outflows compress the gas into a blast wave terminated by
a leading shock front, which  moves outwards with a lower but still
supersonic speed and sweeps out the surrounding medium. Eventually,
this medium is expelled from the galaxy. The model follows in detail the expansion of the 
blast wave through the galaxy disc, and computes the fraction of gas expelled from the galaxy.  
These depend on the ratio $\Delta E/E$ between the  energy injected into the galactic gas 
(taken to be proportional to the energy radiated by the 
AGN through the efficiency {\color{red} $\epsilon_{AGN}=5\times 10^{-2}$) }
and the thermal energy of the unperturbed gas (see Menci et al. (2008)~\cite{Menci2008} for details). 

\textbf{Linear fitting and scatter comparison}.  
%How the fitting is done. The comparing of the histogram and the KS test. The scatter. Figures.
To quantify the agreement between the simulation and the observation, we use a linear regression to fit their relations and compare the result. Our selection window has a hard cut on the \mbh\ value (i.e., vertical direction in Figure~\ref{fig:selectfunc}), and thus the scatter on the host properties are larger (horizontal direction). Thus, we fit the host properties (i.e., \mstar) as a function of \mbh. We adopt the {\sc Scipy} package to estimate the best-fit inference and $1-\sigma$ confidence interval for the simulated sample. We then use the sample slope value to fit for the observations. The comparison results are showing in Figure~\ref{fig:MM_comp}. To estimate the scatter of the sample, we plot the histogram of the residual distribution, i.e., the sample deviation to their best-fit. The result is presented in Figure~\ref{fig:offset_comp} (left panel). We find the standard derivation of the residual for observed and \mbii\ sample are both equal to $\sim0.3$~dex; however, for \sam, the scatter value is much larger ($\sim0.7$~dex).  We perform the KS test of the scatter distribution between observed and \mbii\ sample and infer the p-value as $\sim0.3$. Considering that the simulation sample has been processed to have the same uncertainty level and selection effect, we expect the \mbii\ sample and the observational sample have the same intrinsic scatter level. We adopt the python package {\sc Linmix} to estimate this intrinsic scatter based on the \mbii\ overall sample and obtain the level as $0.25$~dex.

We also present the comparison of the \mbh-\mr\ relations in Figure~\ref{fig:ML_comp} and the comparison of the scatter in Figure~\ref{fig:offset_comp} (right panel). The results are similar to \mbh-\mstar\ relations.

\begin{figure*}[t]%[!b]
\begin{tabular}{c c}
\includegraphics[trim = 0mm 0mm 61mm 0mm, clip, width=0.47\linewidth]{MBII_ML.pdf} &
\includegraphics[trim = 0mm 0mm 61mm 0mm, clip, width=0.47\linewidth]{SAM_ML_consider_nois.pdf} \\
\end{tabular}
\caption{Same as the Figure~\ref{fig:MM_comp}, but for \mbh-\mr\ relation.}
\label{fig:ML_comp}
\end{figure*}


\begin{figure*}[t]%[!b]
\begin{tabular}{c c}
\includegraphics[width=0.45\linewidth]{comp_scatter_ML.pdf} &
\includegraphics[width=0.45\linewidth]{comp_scatter_MM.pdf} \\
\end{tabular}
\caption{The histogram of the scatter (i.e., residuals in the linear relation). The standard derivations for these distribution are $\sim0.3$~dex, $\sim0.3$~dex and $\sim0.7$~dex for observed sample, \mbii\ sample and \sam\ sample, respectively, for both \mbh-\mstar\ and \mbh-\mr\ relations.
}
\label{fig:offset_comp}
\end{figure*}

%%%%%%%%%%%%%%%%%%%%%%%%%%%%%%%%%%%%%%%%%%%%%%%%%%%%%

\section*{References}
\bibliography{references} 


\begin{addendum}
 \item[Acknowledgements] 
Based in part on observations made with the NASA/ESA Hubble Space Telescope, obtained at the Space Telescope Science Institute, which is operated by the Association of Universities for Research in Astronomy, Inc., under NASA contract NAS 5-26555. These observations are associated with programs \#15115. Support for this work was provided by NASA through grant number HST-GO-15115 from the Space Telescope Science Institute, which is operated by AURA, Inc., under NASA contract NAS 5-26555. The authors fully appreciate input from Simon Birrer, Matthew A. Malkan. XD and TT acknowledge support by the Packard Foundation through a Packard Research fellowship to TT. JS is supported by JSPS KAKENHI Grant Number JP18H01251 and the World Premier International Research Center Initiative (WPI), MEXT, Japan. A.S. is supported by the EACOA fellowship.
\ding{More acknowledge.}

%
\item[Author Contributions] X.D. carried out the \hst\ data analysis, led the comparison to the simulating models and was the principal author of the paper. T.T. and J.S. designed the \hst\ project and selected the AGN sample. A.B. and T.D. ran the \mbii\ simulation and provide the simulation data. N.M. ran the \sam\ model and provide and the \sam\ data. X.D, T.T, J.S., A.B., T.D., and N.M. interpreted the result.

\end{addendum}

%%%%%%%%%%%%%%%%%%%%%%%%%%%%%%%%%%%%%%%%%%%%%%%%%%%%%%%%%%%%%%%%%%%%%%%%%%%%%%%
\section*{Additional information}
%\textbf{Code availability.} The \lenstronomy, which is used to decompose the AGN image  
%The data reduction package used to process the \sam\ data is available at http://ascl.net/1407.006, and makes use of 2dfdr: http://www.aao.gov.au/science/software/2dfdr. To derive the stellar kinematic parameters and the lick absorption line strengths, we use the publicly available penalised pixel-fitting (pPXF) code from M. Capppellari: {http://www-astro.physics.ox.ac.uk/~mxc/software/\#ppxf}. For the adaptive LOESS smoothing, we use the code from M. Cappellari obtained from: http://www-astro.physics.ox.ac.uk/~mxc/software/\#loess

\textbf{Data availability.} All the inference of the AGN properties are presented in the companion paper.
%All reduced data-cubes in the GAMA fields used in this Letter are available on: http://datacentral.aao.gov.au/asvo/surveys/sami/, as part of the first SAMI Galaxy Survey data release. Stellar kinematic data products will become available in the second SAMI Galaxy Survey data release. 

\textbf{Correspondence and requests for materials} should be addressed to X.D.~(email:dxh@astro.ucla.edu).

%%%%%%%%%%%%%%%%%%%%%%%%%%%%%%%%%%%%%%%%%%%%%%%%%%%%%%%%%%%%%%%%%%%%%%%%%%%%%%%


%\clearpage
%\newpage
%\onecolumn
%\begin{center}
%{\bf \Large \uppercase{Supplementary information} }
%\end{center}
%
%\setcounter{figure}{0}
%\vspace{2cm}
%
%\begin{figure}[!h]
%\begin{center}
%\includegraphics[width=0.8\linewidth]{hst_sample_bhmf.pdf}
%\caption{
%The selection function of our observation data. Eddington ratios (LBol/LEdd) and BH masses (bottom panel) of our sample (in color) that fall well-below the knee of the BH mass function at z = 1.5 (top panel; Schulze et al. 2015). Dashed lines (vertical and horizontal) denote our selection window with the slanted line only shown to approximately illustrate the effect of a luminosity limit, inherent in the parent catalogs. For reference, we indicate the high-z luminous SDSS QSO samples (grey squares - Peng et al. 2006; grey circles - Decarli et al. 2010) with all falling above our chosen upper mass limit.
%}
%\label{fig:support}
%\end{center}
%\end{figure}

\end{document}
