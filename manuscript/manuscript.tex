\documentclass[apj]{emulateapj}

\usepackage{times}
\usepackage{natbib}
\usepackage[backref,breaklinks,colorlinks,citecolor=cyan]{hyperref} 
\usepackage{booktabs}
\usepackage{graphicx}
\usepackage{bm}
%\usepackage{deluxetable}
\usepackage{multirow}
\usepackage{enumitem}
\usepackage{amsmath}
\usepackage{float}
\usepackage[caption=false]{subfig}

\shorttitle{Cosmic evolution of the \mbh-host relation}
\shortauthors{X. Ding et al.}

\begin{document}
\def\lcdm{$\Lambda$CDM}
\def\hst{{\it HST}}
\def\efr{$R_{\mathrm{eff}}$}
\def\galfit{{\sc Galfit}}
\def\mbh{$\mathcal M_{\rm BH}$}
\def\lhost{$L_{\rm host}$}
\def\jcap{Journal of Cosmology and Astroparticle Physics}
\def\halpha{${\it H}\alpha$}
\def\hbeta{${\it H}\beta$}
\def\sersic{S\'ersic}
\def\lenstronomy{{\sc Lenstronomy}}
\def\Reff{{$R_{\mathrm{eff}}$}}
\def\kms{km~s$^{\rm -1}$}
\def\sigstar{{$\sigma_*$}}
\def\smass{{$M_*$}}
\newcommand{\Mgii}{Mg$_{\rm II}$}
\newcommand{\Civ}{C$_{\rm IV}$}
\newcommand{\sint}{$\sigma_{\rm int}$}
\newcommand{\angstrom}{\text{\normalfont\AA}}
\makeatletter
\newcommand\footnoteref[1]{\protected@xdef\@thefnmark{\ref{#1}}\@footnotemark}  %For share same footnote
\makeatother

\title{The mass relations between supermassive black holes and their host galaxies at $1< z<2$ with \hst-WFC3}

\author{Xuheng Ding\altaffilmark{1, 2}, John Silverman\altaffilmark{3}, Tommaso Treu\altaffilmark{1}, Andreas Schulze\altaffilmark{6}, Malte Schramm\altaffilmark{6}, Simon Birrer\altaffilmark{1}, Daeseong Park{7}, Knud Jahnke, Federica Duras\altaffilmark{4, 5}, Angela Bongiorno\altaffilmark{5}, et al.
 }
 \email{dxh@astro.ucla.edu}
\altaffiltext{1}{Department of Physics and Astronomy, University of California, Los Angeles, CA, 90095-1547, USA}
\altaffiltext{2}{School of Physics and Technology, Wuhan University, Wuhan 430072, China}
\altaffiltext{3}{Kavli Institute for the Physics and Mathematics of the Universe, The University of Tokyo, Kashiwa, Japan 277-8583 (Kavli IPMU, WPI)}
\altaffiltext{4}{Dipartimento di Matematica e Fisica, Universit� Roma Tre, via della Vasca Navale 84, I-00146, Roma, Italy}
\altaffiltext{5}{INAF Osservatorio Astronomico di Roma, via Frascati 33, 00040 Monteporzio Catone, Italy}
\altaffiltext{6}{National Astronomical Observatory of Japan, Mitaka, Tokyo 181-8588, Japan}
\altaffiltext{7}{\textcolor{blue}{To be updated}}
\begin{abstract}

The correlations between the mass of a supermassive black hole (\mbh) and its host galaxy properties (e.g. stellar mass \smass, luminosity \lhost) suggests an evolutionary connection. A powerful test of this co-evolution hypothesis is to measure the relations \mbh-\lhost\ and \mbh-\smass~(total) at high redshift and compare with the local relation. For this purpose, we obtained multi-band {\it Hubble Space Telescope} imaging with WFC3 of 32 X-ray-selected broad-line (type-1) AGN at $1.2<z<1.7$ in deep survey fields. By applying state-of-the-art tools to decompose the \hst\ emission, we obtained accurate measurements of the host galaxy luminosity and stellar mass. The black hole mass (\mbh) is based on near-infrared spectroscopic observations of the broad \halpha~line, which avoids potential uncertainties associated with \Mgii\ or \Civ\ .
%We enlarge the sample using previously published measurements and compare their \mbh-\smass~and \mbh-\lhost~relations to the local to infer their evolution. 
We find that the observed ratio of \mbh/\smass ~is 0.0035, a 0.24~dex offset as compared to the local relation that would lead to a evolution as \mbh/\smass$\propto(1+z)^\gamma$ with $\gamma = 0.80\pm0.30$. However, we expect an offset of 0.21 dex at these redshifts due to a selection bias thus the high-$z$ sample is consistent with the local sample if considering this effect. This is in agreement with the lack of evolution seen in the \mbh/\lhost ~ratios of our sample. Therefore, our results depict a scenario where SMBHs and their host galaxies co-evolve with the possibility for SMBHs to have slightly elevated masses, considering the observed ratios, that may indicate that galaxies with lower angular momentum at high-$z$ may be a bit more efficient to grow their black holes at earlier cosmic times. In any case, the bulge \smass ~mass at a given black hole mass is clearly discrepant with the local relation, since the hosts have a significant amount of stellar mass in a disk, thus leaving the open question as to whether the connection to the bulge is relevant for black hole growth.

\end{abstract}

\keywords{galaxies: active --- galaxies: evolution}

\section{Introduction}
\label{sec:introduction}
%\citep[e.g.,][]{Park15,Kormendy13}

Most galactic nuclei are thought to harbor a supermassive black hole (SMBH), whose mass (\mbh) is known to be correlated with the host properties, such as luminosity (\lhost), stellar mass (\smass) and stellar velocity dispersion (\sigstar). These tight correlations may indicate a connection between nuclear activity, and galaxy formation and evolution~\citep[e.g.,][]{Mag++98, F+M00, M+H03, Gul++09,Beifi2012, H+R04, Geb++01b, Gra++2011}.
Currently, the physical mechanism that can produce such a tight relationship is unknown, due to the daunting range of scales between the dynamical sphere ($\sim$pc) of the SMBH and their host galaxy ($\sim10$ kpc). On one hand, cosmological simulations of structure formation are able to reproduce the mean local correlations considering {\color{blue} either the active galactic nucleus (AGN) feedback as the physical connection~\citep{Springel2005, Hopkins2008, Matteo2008, DeG++15} or how the growth of these components is connected to the common gas supply~\citep{Cen2015, Menci2016}}.
On the other hand, \citet{Peng2007, Jahnke2011, Hirschmann2010} show that, without the need of a physical coupling, another possibility is a statistical convergence from galaxy assembly (i.e., mergers) which can reproduce the observed correlations.

A powerful way to understand the origin of these correlations is to study them as a function of redshift, determining how and when they emerged and evolved over cosmic time~\citep[e.g.,][]{TMB04,Sal++06,Woo++06, Jah++09,SS13,Sun2015}. During the past decade, there has been much progress on this front at $z<1$ using type-1 AGNs. For example, the work by \citet{Park15, Tre++07, Pen++06qsob} demonstrated that \mbh\ at fixed mass resided in less luminous galaxies than today. Similarly, the works by \citet{Bennert11, Woo++08} find the positive evolution of \mbh\ in the observation of the \mbh-\smass\ and \mbh-\sigstar\ relations. Of similar results, \citet{Merloni2010} decomposed the entire spectral energy distributions (SED) into a nuclear AGN and host-galaxy components and found a positive evolution of the mass ratios of black holes to their host galaxies; although, the accuracy of such an approach has not yet been well established. If realized, such offsets can be interpreted as a scenario in which SMBHs were built up first with galaxies then growing around their deep potential wells. 

However, there are studies \citep{Cisternas2011,SS13,Mechtley2012} based on \hst\  imaging of deep survey fields such as COSMOS and CDFS that report no evolution in the \mbh - \smass ~ mass ratio as compared to the local relation. In support, \citet{Sun2015} has re-analyzed the mass ratios for BLAGN in the COSMOS field in a similar manner (i.e., stellar mass measurements from SED fitting) to \citep{Merloni2010} and finds no evolution when accounting for selection effects \citep{Schulze2014} that consider the black hole mass functions and Eddington rate distribution. As highlighted below, samples based on 2D image analysis using \hst\ at $z>1$ are limited thus the infrared capabilities of \hst/WFC3 have not been fully exploited to date on this topic.  

In order to make progress, it is important to reduce as much as possible the uncertainties and take great care of selection effects and systematic errors. First, one needs to deal with the inherent uncertainties in BH mass estimates using the so-called ``virial" method. In particular, many studies rely on \mbh\ estimates using the \Civ\ (or \Mgii) line that may have unknown systematics, such as non-gravitational component of the BLR gas dynamics, when compared to local samples with masses based on broad Balmer lines~\citep[i.e. \halpha\ and \hbeta,][]{Schulze2018, Baskin2005, Trakhtenbrot2012}. Second, since the host information is swamped by the bright nuclear light, measuring the host galaxy properties is challenging. Great care is required for modeling the point spread function and whenever possible it is beneficial to study lensed AGNs, since lensing magnifies the host image to lensed arcs~\citep{Pen++06qsob, Ding2017a, Ding2017b}. Last but not least, the selection function needs to be taken into account when interpreting the observations~\citep{Treu2007, Lauer2007}. For instance, it has been demonstrated~\citep{Schulze2011, Schulze2014} that selecting bright AGNs at high redshift results in display steeper slopes than random ones, suggesting the selecting effects can exhibit faster evolution than a random sample. It is also important to consider the selection function when comparing the observed scaling relations with simulated ones~\citep{DeG++15}.

%~\citep{DeG++15}.
%We aim to determine whether AGNs have begun to couple to their host galaxy at $1.2<z<1.7$, an epoch close to the peak in the accretion history of SMBHs~\citep[e.g.,][]{Aird2015}. 
 
In this study, we aim to make progress by utilizing a large sample with high-quality data both for the measurement of the \mbh\ and their host properties, extending to the highest redshifts where evolutionary effects should be strongest. In particular, we measure the properties of 32 host galaxies in redshift range $1.2<z<1.7$ using \hst\ imaging data, and estimate their \mbh\ based on the robust \halpha\ ~detections, using the multi-object spectrograph Subaru/FMOS. Given the high-quality and sample size of our data, we are capable of testing whether the growth of BH predates that of the host by a factor of at least $1.7$ \citep[i.e. $\sim0.23$ dex,][]{Schulze2014}. The paper is organized as follows. We briefly describe the sample selection and the BH mass of the sample in Section~\ref{sec:data}. We describe the observation and collect the PSF library in Section~\ref{observation}. In Section~\ref{sec:analysis} we decompose our sample and study the host galaxy surface photometry. In Section~\ref{sec:result}, we use the multi-band host magnitudes to infer their stellar population, which we applied to derive the rest-frame R band \lhost\ and \smass\ and compared to the local samples. Discussion and conclusion are presented in Section~\ref{sec:dis} and Section~\ref{sec:sum}. Throughout this paper, we adopt a standard concordance cosmology with $H_0= 70$ km s$^{-1}$ Mpc$^{-1}$, $\Omega{_m} = 0.30$, and $\Omega{_\Lambda} = 0.70$. Magnitudes are given in the AB system.

%\section{Observations and data reduction}
%\label{sec:data}
%In this section, we summarize the sample selection, observations, and data reduction of our sample. 

\section{Experimental design}
\label{sec:data}
%Our study overcomes the limitations of previous studies in the following way. First, to minimize selection bias, we select AGNs that fall below the knee of the black hole mass function. Second, we estimate \mbh\ using Balmer lines which avoids potential systematic uncertainties in the UV-based estimators. Third, our X-ray selected sample has lower nuclear-to-host ratios which facilitate the galaxy mass measurements. Moreover, 21/32 systems in our sample have two-band \hst\ data (i.e. WFC3+ACS), whose multi-band information provides reliable K-correction and stellar mass inference. 

We construct a sample of broad-line (FWHM$>2000$ km s$^{-1}$; type-1) AGNs that meet the following criterion to overcome limitations of previous studies.

\begin{itemize}

\item Black hole mass estimates (\mbh) are based on Balmer lines (i.e., H$\alpha$) which avoids potential systematic uncertainties in UV-based estimators \citep{Greene2005}.

\item Black holes masses fall below the knee of the black hole mass function to minimize the selection biases.

\item Eddington ratios typically above 1\% for most of the sample.

\item A X-ray selected sample demonstrated to have lower nuclear-to-host ratios thus facilitating the galaxy mass measurements.

\item \hst/WFC3 imaging of the host galaxy above the 4000~\angstrom\ break and not including the broad \halpha\ line (6563~\angstrom).

\item The large fraction of the sample has addition \hst\ imaging (i.e., \hst/ACS, providing extra color information to achieve reliable K-corrections and stellar mass determinations. 
\end{itemize}

\subsection{Target selection}

The AGN sample is initially detected by the X-ray observations of the COSMOS~\citep{Civano2016}, (E)-CDFS-S~\citep{Lehmer2005, Xue2011}, and SXDS~\citep{Ueda2008} fields. In most cases, the X-ray sources are first identified as broad-line AGNs through optical spectroscopic campaigns with the VLT, Keck and Magellan. Follow-up near-infrared spectroscopic observations of the AGNs in these fields are carried out with Subaru's Fiber Multi-Object Spectrograph~\citep[FMOS, ][]{Kimura2010, Nobuta2012,Matsuoka2013}, covering the wavelength range 0.9-1.8~$\mu m$ which provides the favorable \halpha\ and \hbeta\ lines out to $z\sim1.7$ to estimates the \mbh. More recently, \citet{Schulze2018} present near-IR spectroscopy of a large compilation of 243 X-ray AGN in these fields. It is from this catalog that we primarily select our targets. We refer the reader to that paper for the details of continuum fitting and emission-line modeling. 

Using the \mbh\ estimates as described below (Section~\ref{mbh}), we select targets with masses between $7.5 \lesssim {\rm log}$~\mbh$\lesssim8.5$. The bolometric luminosities presented in \citet{Schulze2018} are used to calculate their Eddington ratio $\lambda = L_{\rm bol}/L_{\rm Edd}$. We are able to select the sample which have higher Eddington ratios when \mbh\ is lower, to assure the AGN is still activate, as shown in Figure~\ref{fig:selection}. Moreover, we have higher preference to select targets which have rest-frame UV images as provided by \citet{Scoville2007}. We list all the AGN systems analyzed in this work (Table~\ref{tab:objlist}).

%\subsection{Black hole mass and Eddington ratios}
\begin{figure}
\centering
{
\includegraphics[height=0.5\textwidth]{fig/AGN_selection.pdf}
}
\caption{\label{fig:selection} 
We present the selection window as used to choose our AGN sample, which is based on the Eddington ratios ($\lambda = L_{\rm Bol}/L_{\rm Edd}$) and \mbh. Our sample (color coded) falls well below the knee of BH mass function at $z=1.5$ \citep[top panel,][]{Schulze2015}. We also control the sample to have higher Eddington ratios when their \mbh\ is lower, to guarantee the AGN activity. For comparing propose, we plot the high-$z$ luminous SDSS QSO samples that has been studied based on the observation of \hst\ in the reference~\citep[grey squares and circles from][respectively]{Peng2006a, Decarli2010}. 
}
\end{figure} 

\begin{deluxetable*}{llccccc}
\tablecolumns{7}
\tablewidth{0pt}
\tablecaption{Details of Observation} 
\tablehead{ 
\colhead{Object ID} &
\colhead{$z$} & 
\colhead{WFC3/Filter} &
\colhead{$RA$}&
\colhead{$DEC$}&
\colhead{Observing date}&
\colhead{Exposure time (s)}
\\ 
\colhead{(1)} &
\colhead{(2)} &
\colhead{(3)} &
\colhead{(4)} &
\colhead{(5)} &
\colhead{(6)} &
\colhead{(7)}
} 
\startdata
%\multicolumn{5}{c}{Sample presented in \citet{Treu+07}}\\
%\\
COSMOS-CID1174 & 1.552 & F140W & 150.2789 & 1.9595 & 2017-10-26 & 2395.4\\
COSMOS-CID1281 & 1.445 & F140W & 150.4160 & 2.5258 & 2018-11-26 & 2395.4\\
COSMOS-CID206 & 1.483 & F140W & 149.8371 & 2.0088 & 2017-10-23 & 2395.4\\
COSMOS-CID216 & 1.567 & F140W & 149.7918 & 1.8729 & 2017-10-23 & 2395.4\\
COSMOS-CID237 & 1.618 & F140W & 149.9916 & 1.7243 & 2018-06-03 & 2395.4\\
COSMOS-CID255\footnoteref{note1} & 1.664 & F140W & 150.1017 & 1.8483 & 2019-03-16 & 2395.4\\
COSMOS-CID3242 & 1.532 & F140W & 149.7113 & 2.1452 & 2017-10-26 & 2395.4\\
COSMOS-CID3570 & 1.244 & F125W & 149.6411 & 2.1076 & 2017-10-27 & 2395.4\\
COSMOS-CID452 & 1.407 & F125W & 150.0045 & 2.2371 & 2017-10-25 & 2395.4\\
COSMOS-CID454 & 1.478 & F140W & 149.8681 & 2.3307 & 2018-02-26 & 2395.4\\
COSMOS-CID50 & 1.239 & F125W & 150.2080 & 2.0833 & 2017-10-23 & 2395.4\\
COSMOS-CID543 & 1.301 & F125W & 150.4519 & 2.1448 & 2018-04-30 & 2395.4\\
COSMOS-CID597 & 1.272 & F125W & 150.5262 & 2.2449 & 2018-11-25 & 2395.4\\
COSMOS-CID607 & 1.294 & F125W & 150.6097 & 2.3231 & 2017-10-25 & 2395.4\\
COSMOS-CID70 & 1.667 & F140W & 150.4051 & 2.2701 & 2017-10-27 & 2395.4\\
COSMOS-LID1273 & 1.617 & F140W & 150.0565 & 1.6275 & 2017-10-31 & 2395.4\\
COSMOS-LID1538 & 1.527 & F140W & 150.6215 & 2.1588 & 2018-05-01 & 2395.4\\
COSMOS-LID360 & 1.579 & F140W & 150.1251 & 2.8617 & 2017-10-30 & 2395.4\\
COSMOS-XID2138 & 1.551 & F140W & 149.7036 & 2.5781 & 2017-11-01 & 2395.4\\
COSMOS-XID2202 & 1.516 & F140W & 150.6530 & 1.9969 & 2017-11-05 & 2395.4\\
COSMOS-XID2396 & 1.600 & F140W & 149.4779 & 2.6425 & 2017-11-12& 2395.4\\
CDFS-1 & 1.630 & F140W & 52.8990 & -27.8600 & 2018-04-03 & 2395.4\\
CDFS-229 & 1.326 & F125W & 53.0680 & -27.6580 & 2018-04-04 & 2395.4\\ 
CDFS-321 & 1.570 & F140W & 53.0486 & -27.6239 & 2018-08-18& 2395.4\\ 
CDFS-724 & 1.337 & F125W & 53.2870 & -27.6940 & 2018-04-05 & 2395.4\\ 
ECDFS-358 & 1.626 & F140W & 53.0850 & -28.0370 & 2018-02-09& 2395.4\\ 
SXDS-X1136 & 1.325 & F125W & 34.8925 & -5.1498 & 2018-01-29& 2395.4\\ 
SXDS-X50 & 1.411 & F125W & 34.0267 & -5.0602 & 2018-03-01& 2395.4\\ 
SXDS-X717 & 1.276 & F125W & 34.5400 & -5.0334 & 2018-07-02 & 2395.4\\ 
SXDS-X735 & 1.447 & F140W & 34.5581 & -4.8781 & 2017-11-14 & 2395.4\\ 
SXDS-X763 & 1.412 & F125W & 34.5849 & -4.7864 & 2018-07-03 & 2395.4\\ 
SXDS-X969 & 1.585 & F140W & 34.7594 & -5.4291 & 2018-07-02 & 2395.4\\ 
\enddata
\label{tab:objlist}
\tablecomments{
Column 1: Object field and ID.
Column 2: Redshifts.
Column 3: WFC3 filter. Note that the targets from the COSMOS field also have ACS/F814W imaging.
Column 4 and 5: J2000 $RA$ and $DEC$ coordinates.
Column 6: The observing start date.
}
\end{deluxetable*}

\subsection{Details on BH mass estimates}
\label{mbh}
The \mbh\ of type-1 AGNs can be inferred using the so-called virial method~\citep{Peterson2004, Shen2013}. The kinematics of the broad-line region (BLR) trace the gravitational field of the central supermassive black hole, assuming the gravity dominates the motion of the BLR gas. In this scenario, the width of the emission-line provided the scale of the velocity, while the AGN continuum luminosity establish an empirical scale of the BLR size. As a result, the estimation of the \mbh\ is achieved by these measurements.

To avoid any systematic bias between our samples and the literature ones, we compared the recipes as introduced by \citet{Schulze2018, Ding2017b}. We find that the \hbeta\ estimator is very consistent (\mbh\ r.m.s. $<0.03$~dex) between the two references. We also find that, in \citet{Schulze2018}, the cross-calibration between the \halpha\ and \hbeta\ has better agreement. Thus, we adopt the virial formalism from \citet{Schulze2018}:

\begin{eqnarray}
\label{eq:Ha}
\log \left(\frac{\mathcal M_{\rm BH} (H\alpha)}{M_{\odot}}\right)&~=~& 6.71+0.48 \log \left(\frac{ \rm L _{H\alpha}}{10^{42}{\rm erg~s^{-1}}}\right) \nonumber\\
&~+~& 2.12 \log \left(\frac{\rm FWHM(H\alpha)}{1000 ~{\rm km~s^{-1}}}\right) ,
\end {eqnarray}

and

\begin{eqnarray}
\label{eq:Hb}
\log \left(\frac{\mathcal M_{\rm BH}(H\beta) }{M_{\odot}}\right)&~=~& 6.91+0.50\log \left(\frac{ \rm L _{\lambda_{5100}}}{10^{44}{\rm erg~s^{-1}}}\right) \nonumber\\
&~+~& 2.0 \log \left(\frac{\rm FWHM(5100)}{1000 ~{\rm km~s^{-1}}}\right) .
\end {eqnarray}

Having defined the recipes, we estimate \mbh\ by adopting the emission line properties measured by \citet{Schulze2018}. {\color{blue} 3/32 systems, including CDFS-1, CDFS-229 and CDFS-724, are not included in \citet{Schulze2018}; their \halpha\ emission lines information is provided by \citet{Suh2015}.
14/32 systems have both \halpha\ and \hbeta\ emission line information. However, we only adopt the value for \mbh\ using the \halpha\ emission line, to assure they are self-consistent.
%Given that 18/32 sample are lack of \hbeta\ line, we thus adopt the \mbh\ estimated by \halpha\ to our sample in the rest of the work, to avoid the scatter of the \mbh\ by \hbeta\ unevenly spreading into the overall sample.

We summarized the inference of the \mbh\ together with the properties of the emission line in Table~\ref{tab:result_mbh}. 
}

\begin{deluxetable*}
{@{\extracolsep{4pt}}lcccccccc}   % need for the gap between the clines.
\tablecolumns{9}
\tablewidth{0pt}
\tablecaption{Inferred BH properties for the 32 systems} 
\tablehead
{ 
\colhead{Target ID}&
  \multicolumn{4}{c}{using emission line \halpha}&
  \multicolumn{4}{c}{using emission line \hbeta} \\
  \cline{2-5}  \cline{6-9} \\
\colhead{}& 
\colhead{FWHM}& \colhead{$\log( \rm L _{H\alpha}$)}& \colhead{$\log$\mbh}& Eddington ratio &
\colhead{FWHM(5100)}& \colhead{$\log( \rm L _{\lambda5100})$}& \colhead{$\log$\mbh}& Eddington ratio \\
\colhead{}& 
\colhead{(\kms)}& \colhead{(${\rm erg~s^{-1}}$)}& 
\colhead{(M$_{\odot}$)}& (log$\lambda$)&
\colhead{(\kms)}& 
\colhead{(${\rm erg~s^{-1}}$)}&\colhead{(M$_{\odot}$)} & (log$\lambda$)\\
\colhead{(1)}& 
\colhead{(2)}& \colhead{(3)}& 
\colhead{(4)}& \colhead{(5)}& 
\colhead{(6)}&\colhead{(7)}&
\colhead{(8)}& \colhead{(9)}
}
\startdata 
CID1174 & 1906 & 43.43 & 7.99 & -0.50 & 5898 & 44.76 & 8.83& -1.35 \\
CID1281 & 1619 & 43.24 & 7.75 & -0.45 & ... & ... & ...& ... \\
CID206 & 3334 & 43.48 & 8.53 & -1.00 & ... & ... & ...& ... \\
CID216 & 2230 & 42.85 & 7.85 & -0.93 & ... & ... & ...& ... \\
CID237 & 2112 & 43.86 & 8.29 & -0.40 & ... & ... & ...& ... \\
CID255 & 1932 & 43.99 & 8.27 & -0.25 & 3709 & 45.37 & 8.73& -0.63 \\
CID3242 & 2543 & 43.83 & 8.45 & -0.58 & 3775 & 45.10 & 8.61& -0.78 \\
CID3570 & 1959 & 43.16 & 7.89 & -0.66 & ... & ... & ...& ... \\
CID452 & 3458 & 42.92 & 8.30 & -1.30 & 3127 & 44.63 & 8.22& -0.88 \\
CID454 & 2824 & 43.34 & 8.31 & -0.91 & ... & ... & ...& ... \\
CID50 & 2340 & 43.94 & 8.42 & -0.45 & 1939 & 45.33 & 8.15& -0.09 \\
CID543 & 2189 & 43.57 & 8.19 & -0.56 & ... & ... & ...& ... \\
CID597 & 1656 & 43.33 & 7.81 & -0.42 & ... & ... & ...& ... \\
CID607 & 3009 & 43.67 & 8.53 & -0.81 & 4242 & 44.78 & 8.56& -1.06 \\
CID70 & 2480 & 43.51 & 8.27 & -0.71 & 3982 & 45.16 & 8.69& -0.80 \\
LID1273 & 3224 & 43.61 & 8.56 & -0.90 & ... & ... & ...& ... \\
LID1538 & 2941 & 43.60 & 8.47 & -0.83 & ... & ... & ...& ... \\
LID360 & 2482 & 43.88 & 8.45 & -0.53 & 2869 & 45.09 & 8.37& -0.55 \\
XID2138 & 3186 & 43.61 & 8.55 & -0.89 & 2945 & 44.81 & 8.25& -0.72 \\
XID2202 & 2973 & 43.56 & 8.46 & -0.85 & ... & ... & ...& ... \\
XID2396 & 2271 & 44.06 & 8.46 & -0.37 & 2658 & 45.50 & 8.51& -0.27 \\
CDFS-1 & 5449 & 43.08 & 8.79 & -0.98 & ... & ... & ...& ... \\
CDFS-229 & 2254 & 43.30 & 8.08 & -0.48 & ... & ... & ...& ... \\
CDFS-321 & 2442 & 43.93 & 8.46 & -0.50 & ... & ... & ...& ... \\
CDFS-724 & 3352 & 42.56 & 8.09 & -1.22 & ... & ... & ...& ... \\
ECDFS-358 & 2237 & 43.40 & 8.12 & -0.67 & ... & ... & ...& ... \\
SXDS-X1136 & 2760 & 43.43 & 8.33 & -0.85 & 6761 & 44.71 & 8.93& -1.50 \\
SXDS-X50 & 1817 & 43.42 & 7.94 & -0.47 & ... & ... & ...& ... \\
SXDS-X717 & 2931 & 43.05 & 8.20 & -1.08 & ... & ... & ...& ... \\
SXDS-X735 & 2702 & 43.70 & 8.44 & -0.70 & 3520 & 45.07 & 8.54& -0.73 \\
SXDS-X763 & 2961 & 43.57 & 8.47 & -0.84 & 4509 & 44.51 & 8.47& -1.27 \\
SXDS-X969 & 2296 & 43.50 & 8.20 & -0.64 & 1696 & 45.05 & 7.90& -0.11 \\
\enddata
\label{tab:result_mbh}
\tablecomments{
Column 1: Object ID.
Column 2-4: \halpha\ emission line width (FWHM), continuum luminosity, and inferred \mbh. 
Column 5-7: \hbeta\ emission line width (FWHM), continuum luminosity, and inferred \mbh. 
The typical uncertainty level for FWHM and $\log(\lambda \rm L)$ are $15\%$ and $\pm0.01$, respectively. The inferred uncertainty level for \mbh\ are assumed as 0.4 dex.
}
\end{deluxetable*}


\subsection{Comparison sample}\label{sec:compare_sample}

We aim to collect the \mbh-\lhost\ and \mbh-\smass\ relations of our samples and comparing to the ones at different redshift. 
For comparison with the local relation, we use the local AGN measurements by \citet{Ben++10} and \citet{Bennert++2011} (hereafter, B10 and B11) to define our zero-point. The sample by B10 consists of 19 \mbh-\lhost\ measured with reliable \mbh\ measured using reverberation-mapped (uncertainty level $\sim0.15$ dex). The work of B11 contains 25 \mbh-\smass\ local active AGNs, where the \mbh\ are measured from the single-epoch method (\mbh\ uncertainty level $\sim0.4$ dex). To increase the local \smass\ and \mbh\ to the higher range, we also include 30 inactive galaxies (mainly ellipticals or S0) as measured by \citet{H+R04} (hereafter HR04).
It is worth noting that in the local comparison sample, the bulge mass is equivalent to the total stellar mass for the local comparison sample. In other words, our local comparison is the \mbh-\smass$_{Bulge}$ and not those involving total quantities.

We include in our analysis published samples at intermediate redshift range to understand the redshift evolution of these correlations. We select samples that were analyzed by members of our team, in order to ensure uniform measurements. The intermediate redshift AGNs that we consider include, for the \mbh-\lhost\ relation, 52 objects published by \citet{Park15} at $0.36<z<0.57$ and 27 objects published by \citet{Bennert11, SS13} at $0.5<z<1.9$. Moreover, \citet{Bennert11, SS13} also derive the \smass\ for the 27 objects which enable us to compare to the \mbh-\smass\ relation. {\color{blue} In addition, we adopt a sample of 32 \mbh-\smass\ measurements at $0.3<z<0.9$ by \citet{Cisternas2011} to compare with our sample.

Across the intermediate redshift sample, we recalibrate the \mbh\ by \halpha\ or \hbeta\ using the self-consistent recipes introduced in Section~\ref{mbh}. For \mbh\ estimated by \Mgii\, we adopt the recipe by \citet{Ding2017b} to recalibrate them.
}

\section{\hst\ observations}
\label{observation}
%\subsubsection{HST image Observation}
% The AGN surface photometry in IR Channel Filters.\\
High spatial resolution imaging is required for the decomposition of the nuclear/host emission and the accurate estimate of the host stellar mass. For this purpose, all the new 32 AGN systems were observed with the \hst/WFC3 infrared channel, as part of the \hst\ program GO-15115, PI: John Silverman. We selected to use the filters F125W $(1.2<z<1.44)$ and F140W $(1.44<z<1.7)$ according to the redshift of the targets. This selection ensures that the broad \halpha\ line is not present in the bandpass so as not to contaminate the host emissions due to the broad wings of the PSF.

%Drizzling, background noise...
For each of the 32 new AGNs we obtained six separate exposures with $\sim399s$ (i.e. total exposure time $\sim2394s$). Table~\ref{tab:objlist} lists the details of the observation. The six exposures for each dither image were combined with the {\sc astrodrizzle} software package, with an output pixel scale of 0\farcs{0642} by setting \texttt{pixfrac} parameter as 0.8 using the \texttt{gaussian} kernel\footnote{\label{note1}For CID255, 3/6 of the dither WFC3 images are corrupted. We achieve to analyze this sample with a same approach, taking the 3 available frames.}. 
To accurately estimate the background light which could come from both the sky and the detector, we adopt the {\sc photutils} tool.

Multi-band information provides the SED at a more precise level. 21/32 objects in our sample have rest-frame UV images by \citet{Scoville2007}. Images were taken through the wide ACS/F814W filter at four dither pointing with $507s$ exposures. The final image is drizzled to 0\farcs{03} pixel scale. Given the multi-band image for these systems, we are able to infer their host color and assess the contribution of both the young and old stellar population %to the stellar mass budget 
which insures an accurate inference of rest-frame R-band luminosity by K-correction, stellar mass and star formation rate (SFR) \citep{Gallazzi2009} . 

%\subsubsection{AGN Emission Line}
%\label{sec:bh_mass}
% The introduction of the BH inferred by the board emission line.

 
%\section{Analysis}
\subsection{PSF library}
\label{sec:psf_library}
%Introduce the PSF library
The knowledge of the PSF is crucial for the AGN imaging decomposition, especially when the point source is bright relative to the host. The PSF is known to be vary across the detector and with time, due to the effects of aberrations and breathing. Simulated PSF, such as {\sc TinyTim} ones, are usually insufficient matches to observations for our purposes \citet{Mechtley2012}. Stars within the fields provide a better description than the simulated PSF, since they were observed simultaneously with the science targets and reduced and analyzed in the same way \citet{Kim2008, Park15}

To minimize the impact of the mismatch, we build a PSF library by selecting all the isolated, unsaturated PSF-stars with high S/N from our entire program. The selection consists of the following steps. We first adopt the identified stars as candidates from the COSMOS2015 catalog by \citet{Laigle2016}. A lot of bright stars with intensity similar to the AGNs were excluded by \citet{Laigle2016}; in the second step, we manually select the PSF-like objects as candidates. We then remove the non-ideal PSF candidates based on their intensity, FWHM, central symmetry and any that were contaminated by nearby objects. In the end, the PSF library contains 78 and 37 PSFs imaged through filters F140W and F125W, respectively.

We assume that the stars in the library are representative of the possible PSFs in our program. Therefore, the PSF library gives a good representation of the dominant source of uncertainty as well as the best fit.
%In the next subsection, we carry out the AGN decomposition using each PSF. We rank their performance based on the goodness (i.e. $\chi^2$) so that the final inference of the host properties are weighted from the top-rank PSFs.



%\label{sec:analysis}

%\subsection{Surface Photometry}
%\subsubsection{PSF library}    
%\subsubsection{Surface Photometry modeling method}
\section{AGN-host Decomposition}
\label{sec:analysis}
%Introduce the lenstronomy. Model the PSF in each library.\\
%Inspired by Simon's work, we rank the performance of the each PSF and weight for the final fitting result. 
We simultaneously fit the two-dimensional flux distribution of the center nuclear and the underlying host galaxy. Following common practice, we assume the unresolved nuclear as a scaled point source, while the host galaxy as a \sersic\ profile. Note that the actual morphologies of the host galaxies could be more complicated (e.g. bulge+disk). However, the purpose of adopting the \sersic\ model is a simplified a first-order approximation of the surface brightness distribution with a flexible parameterization which provides sufficient freedom to infer the host flux, given the high redshift range of our sample. Furthermore, we fit other galaxies that happen to be close to the AGN as additional \sersic\ model, to account for any potential contamination. The systems CID206 and ECDFS-358 have nearby objects which could not be described by \sersic\ model, we thus mask these objects in the fitting.

%Introduce Lenstronomy:
We adopt the imaging modeling tool \lenstronomy~\citep{lenstronomy} to perform the decomposition of the host and nuclear light. \lenstronomy\ is a multi-purpose open-source gravitational lens image modeling python package. 
Its flexibility enables us to turn off the lensing channel and focus on the AGN decomposition\footnote{For a sanity check, we compared the performance of \lenstronomy\ to the most commonly used galaxy modeling software {\sc Galfit} and confirm that the consistent result could be obtained with the same fitting ingredients provided.}. The maim advantage over {\sc Galfit} is that \lenstronomy\ returns the full posterior of the parameter and not just the best fit model and the Laplace approximation of the uncertanties. The input ingredients to \lenstronomy\ include:
\begin{enumerate}
\item AGN imaging data. \\
-- Using aperture photometry, we find that an aperture size with radius $\sim1\farcs{}5$ covers sufficient AGN light our samples. By default, we cut out the AGN image to a $61\times61$ pixels box (i.e. $4''\times 4''$). If needed, the larger size box would be sufficient to include the nearby objects. 
\item Noise level map.\\
-- The origin of the noise in each pixel stems from the read noise, background noise and Poisson noise by the astronomical sources themselves. To take them into account, we measure the read noise and background noise level directly from the empty regions of the data. We calculate the effective exposure time of each pixel based on the drizzled \texttt{WHT} array maps to infer the Poisson noise level. 
\item PSF. \\
-- 
The PSF is directly taken from the PSF library. Usually, a mismatch exists when subtracting the AGN as the scaled PSF, especially at the central parts. While modeling multiply imaged AGN, this mismatch can be mitigated with PSF reconstruction by the iterative method~\citep{Chen2016, Birrer2018}.  However, this approach requires multiple images which are not available in our case.  We remedy this deficiency by using a broad library which should contain sufficient information to cover all the possible PSF. 
\end{enumerate}

%Introduce the final inference using fitting:
The host properties of one AGN system is inferred via the following steps. First, we model the AGN and host using each PSFs in the library. Inputting the fitting ingredients to \lenstronomy, the posterior distribution of the parameter space is calculated and optimized by adopting the Particle Swarm Optimizer (PSO)\footnote{Note that, the \lenstronomy\ enables one to further infer the parametric confidence interval using MCMC. In our case, given a fixed PSF, the $1-\sigma$ inference of each parameter are extremely narrow. Thus, we only take the best-fit inference using PSO for the further calculations.}.
To avoid any unphysical inferences, we set the upper and lower limit for the key parameters as effective radius \Reff\ $\in[0\farcs{}1,1\farcs{0}]$, and \sersic\ index $n\in[1,7]$\footnote{{\color{blue}
%As will see, three AGN systems would have finally inference with abnormally high \sersic\ index value (i.e., $n>5$). We test to reanalyze these sample and fix the \sersic\ index within $n\in[1,4]$ and find their results changes very little ---  $\Delta$\lhost\ and $\Delta$\smass\ within 0.1 dex.
We test to use $n\in[1,4]$ as prior to fit the systemic with high \sersic\ index ($n>4.5$), and find that the changes on the inference of their host magnitude are very limited ($<0.03$ dex).}
}.
Then, we rank the performance of each PSF based on the $\chi^2$ value and select the top-eight PSFs as representative of the best fit PSFs. We infer the host \sersic\ parameters (i.e., host flux, \Reff, \sersic\ index) using a weighted arithmetic mean. The weight is calculated by:
\begin{eqnarray}
\label{eq:weights}
w_i = exp \big(- \alpha \frac{ (\chi_i ^2 - \chi_{best} ^2 )}{2 \chi_{best} ^2} \big),
\end{eqnarray} 
where the $\alpha$ is an inflation parameter so that when $i=8$:
\begin{eqnarray}
\label{eq:alpha}
\alpha \frac{ \chi_{i=8} ^2 - \chi_{best} ^2 }{2 \chi_{best} ^2} = 2,
\end{eqnarray} 
The goal of this recipe is to weight each PSF based on their relative goodness of fit, while ensuring at least eight are used to capture the range of systematic uncertainties. The answers would not change significantly if we had chosen a different number of PSFs, as we show below.

Note that since each AGN system was observed in different location of the detector and at a different time, the top-eight PSFs usually vary from one AGN system to another. Given the weights, the inferred value of host properties and the root-mean-square ($\sigma$) are calculated as:
\begin{eqnarray}
\label{eq:infer_value}
\bar{x}  =  \frac{  \sum_{i=1}^{8}   x_i * w_i  }{\Sigma w_i} ,
\end{eqnarray} 
\begin{eqnarray}
\sigma =   \sqrt{ \frac{  \sum_{i=1}^{8}   (x_i -  \bar{x} ) ^2 * w_i  }{\Sigma w_i} }.
\end{eqnarray} 

In Figure~\ref{fig:AGN_decomp}, we demonstrate the optimized results inferred by \lenstronomy, including the AGN images, best-fit models, image subtract point source (i.e. host) and the residuals, taking COSMOS-CID1174 as an example. The weights adopted in the analysis are listed in Table~\ref{tab:weight_CID1174}. In order to quickly overview our inference, we summarize the relations of the properties between effective radius \Reff, 2) \sersic\ index and 3) host to total flux ratio in Figure~\ref{fig:flux_r_n_corner}. We also comparing the distribution of the \sersic\ values to the inactive galaxies in the Appendix~\ref{sec:comp_inactive}.

\begin{figure*}
\centering
%\hspace{-5.5em}
{
\includegraphics[height=0.25\textwidth]{fig/best_fit_CID1174_SB_profile.pdf}
\caption{\label{fig:AGN_decomp} 
Figure illustrates the inference for the 32 objects based on the WFC3 images.
In each row, observed data (first column), best-fit models (second column), data subtracted by the inferred point source (third column),  normalized residuals (fourth column) are presented together with the target ID. In the fifth column, we present the 1-D surface brightness profiles (top) and the corresponding residual (bottom). The 1-D profiles indicate the surface brightness including the data (open circles), the best-fit model (blue line), the AGN (orange line), and the model for the extended sources (green line, i.e., host and other objects). Note that the one-dimensional surface brightness profiles are only for illustration purposes. The actual fitting is based on the two-dimensional images. The remaining 31 objects of the sample are shown in Appendix~\ref{sec:restsample}.
}}
\end{figure*} 

\begin{deluxetable*}{ccccccc}
\tablecolumns{7}
\tablewidth{0pt}
\tablecaption{Host inference of CID1174} 
\tablehead{ 
\colhead{PSF rank} &
\colhead{total $\chi ^2$} & 
\colhead{weights $w_i$} &
\colhead{host flux (counts)} &
\colhead{host flux ratio}&
\colhead{\Reff (arcsec)}&
\colhead{\sersic\ $n$}
 \\ 
\colhead{(1)} &
\colhead{(2)} &
\colhead{(3)} &
\colhead{(4)} &
\colhead{(5)} &
\colhead{(6)} &
\colhead{(7)} 
} 
\startdata
%\multicolumn{5}{c}{Sample presented in \citet{Treu+07}}\\
%\\
1 & $8584.429$ & $1.000$ & $82.2$ & $35\%$ & $0\farcs{}345$ & $1.1$ \\
2 & $8646.711$ & $0.920$ & $99.1$ & $42\%$ & $0\farcs{}298$ & $1.9$ \\
3 & $8816.947$ & $0.734$ & $76.7$ & $33\%$ & $0\farcs{}365$ & $1.1$ \\
4 & $9304.841$ & $0.383$ & $128.6$ & $55\%$ & $0\farcs{}231$ & $2.8$ \\
5 & $9652.575$ & $0.241$ & $187.5$ & $79\%$ & $0\farcs{}116$ & $6.2$ \\
6 & $9917.101$ & $0.170$ & $100.2$ & $42\%$ & $0\farcs{}287$ & $2.1$ \\
7 & $10018.324$ & $0.148$ & $75.1$ & $32\%$ & $0\farcs{}365$ & $1.2$ \\
8 & $10087.456$ & $0.135$ & $79.8$ & $34\%$ & $0\farcs{}358$ & $1.2$ \\
\hline\\
Weighted value & & & $97.322\pm28.336$ & $42\%\pm12\%$& $0\farcs{}309\pm0\farcs{}065$  & $1.9\pm1.3$  \\
\enddata
\label{tab:weight_CID1174}
\tablecomments{
Column~1: Rank of the PSF from the library.
Column~2: Total $\chi ^2$ for the corresponding PSF.
Column~3: Weights for the inference.
Column~4-7: Fitted value for the host flux, host/total flux ratio, effective radius, and \sersic\ index.
For this sample, the inflation parameter $\alpha$ calculated by Eq.~\ref{eq:alpha} is 16.671.
}
\end{deluxetable*}

\begin{figure*}
\centering
{
\includegraphics[height=0.75\textwidth]{fig/flux_r_n_corner.pdf}
}
\caption{\label{fig:flux_r_n_corner}
Illustration of the relations between the inferred host \sersic\ parameters.}
\end{figure*} 

The inference of the host properties is weighted by eight top-ranked PSFs. The smaller volume of top-ranked  PSFs would possibly underestimate the uncertainties of the inference. To understand how the choice of the number of top-ranked PSF affects our inference, we compare to the inference by using five and ten top-ranked PSFs. As shown in Figure~\ref{fig:hist_compare}, the results are very consistent, meaning that the host inference has been solid when the amount of selected PSF exceeds five. 

\begin{figure*}
\centering
{
\includegraphics[height=0.35\textwidth]{fig/hist_compare.pdf}
}
\caption{\label{fig:hist_compare} 
Comparison of the histogram of the inferred host flux ratio, based on the different volum of the top-ranked PSFs.}
\end{figure*} 


%The inference of F814w data.
18/32 systems in the COSMOS field have ACS/F184W imaging data. We infer their host flux ratio using the same approach used for the WFC3-IR. The field of view by ACS are larger than WFC3, hence we collected in total 174 ACS PSFs in the library. In principle, the host inference by IR band is superior to the one by UV band, giving that the effects of dust extinction and contrast between the (blue) AGN and (red) host. Thus, we fix the \Reff\ and \sersic\ $n$ as the value inferred by IR band, assuming that the morphology of the galaxy is consistent between ACS and WFC3.

We list all the inference of the host galaxy properties in the Table~\ref{tab:result_sersic}.

\tabcolsep=0.03cm
\begin{deluxetable*}
{@{\extracolsep{4pt}}lcccccccccc}   % need for the gap between the clines.
\tablecolumns{9}
\tablewidth{0pt}
\tablecaption{Inferred host galaxy properties for the 32 systems.} 
\tablehead
{ 
\colhead{Target ID}&
  \multicolumn{5}{c}{WFC3}&
  \multicolumn{3}{c}{ACS/F814W} &
   \multicolumn{2}{c}{host properties} \\
  \cline{2-6}  \cline{7-9} \cline{10-11}  \\
\colhead{}& 
\colhead{reduced $\chi ^2$}& \colhead{host-total flux ratio}& 
\colhead{\Reff}& \colhead{\sersic\ $n$}& 
\colhead{magnitude}&
\colhead{reduced $\chi ^2$}& \colhead{host-total flux ratio}& \colhead{magnitude} &
\colhead{$\log L_R$} &\colhead{$\log M_*$} \\
\colhead{}& 
\colhead{}& \colhead{}& 
\colhead{($\arcsec$)}& \colhead{}& 
\colhead{(AB system)}& \colhead{}& 
\colhead{}& \colhead{(AB system)} &\colhead{($L_{\odot,R}$)} & \colhead{(M$_{\odot}$)}\\
\colhead{(1)}& 
\colhead{(2)}& \colhead{(3)}& 
\colhead{(4)}& \colhead{(5)}& 
\colhead{(6)}& \colhead{(7)}& 
\colhead{(8)}& \colhead{(9)}&
\colhead{(10)}& \colhead{(11)}
}
\setlength{\tabcolsep}{20pt}
\renewcommand{\arraystretch}{1.5}
\startdata 
CID1174 & $2.307$ & $42\%\pm12\%$ & $0\farcs{}31\pm0\farcs{}07$ & $1.9\pm1.3$ & $21.48\substack{+0.37\\-0.28}$ & $2.496$ & $11\%\pm1\%$ & $23.21\substack{+0.11\\-0.10}$ & $11.05\substack{+0.15\\-0.11}$ &$10.78\substack{+0.18\\-0.15}$\\[3pt]
CID1281 & $1.322$ & $49\%\pm14\%$ & $0\farcs{}24\pm0\farcs{}09$ & $3.2\pm1.5$ & $22.88\substack{+0.36\\-0.27}$ & $1.378$ & $19\%\pm8\%$ & $24.83\substack{+0.60\\-0.38}$ & $10.41\substack{+0.15\\-0.11}$ &$10.14\substack{+0.18\\-0.15}$\\[3pt]
CID206 & $2.054$ & $35\%\pm24\%$ & $0\farcs{}29\pm0\farcs{}15$ & $3.1\pm2.5$ & $21.82\substack{+1.30\\-0.58}$ & $1.903$ & $8\%\pm2\%$ & $23.67\substack{+0.40\\-0.29}$ & $10.86\substack{+0.52\\-0.23}$ &$10.59\substack{+0.53\\-0.25}$\\[3pt]
CID216 & $1.514$ & $94\%\pm5\%$ & $0\farcs{}25\pm0\farcs{}06$ & $6.2\pm1.2$ & $21.51\substack{+0.05\\-0.05}$ & $1.425$ & $35\%\pm2\%$ & $23.45\substack{+0.05\\-0.05}$ & $11.05\substack{+0.03\\-0.03}$ &$10.78\substack{+0.10\\-0.10}$\\[3pt]
CID237 & $2.349$ & $30\%\pm6\%$ & $0\farcs{}87\pm0\farcs{}17$ & $4.7\pm1.7$ & $21.28\substack{+0.26\\-0.21}$ & $2.354$ & $3\%\pm2\%$ & $23.72\substack{+1.04\\-0.52}$ & $11.18\substack{+0.11\\-0.09}$ &$10.91\substack{+0.14\\-0.13}$\\[3pt]
CID255 & $1.625$ & $19\%\pm5\%$ & $0\farcs{}19\pm0\farcs{}06$ & $4.2\pm1.5$ & $21.61\substack{+0.37\\-0.28}$ & 2.858 & $4\%\pm2\%$ & $22.89\substack{+0.60\\-0.39} $& $11.08\substack{+0.15\\-0.11}$ &$10.81\substack{+0.18\\-0.15}$\\[3pt]
CID3242 & $2.751$ & $46\%\pm13\%$ & $0\farcs{}20\pm0\farcs{}16$ & $6.1\pm1.9$ & $21.16\substack{+0.35\\-0.26}$ & $2.596$ & $5\%\pm1\%$ & $23.60\substack{+0.34\\-0.26}$ & $11.16\substack{+0.14\\-0.11}$ &$10.89\substack{+0.17\\-0.15}$\\[3pt]
CID3570 & $1.665$ & $77\%\pm2\%$ & $0\farcs{}70\pm0\farcs{}01$ & $0.7\pm0.1$ & $21.16\substack{+0.02\\-0.02}$ & $1.332$ & $86\%\pm2\%$ & $22.97\substack{+0.01\\-0.01}$ & $10.98\substack{+0.02\\-0.02}$ &$10.71\substack{+0.10\\-0.10}$\\[3pt]
CID452 & $1.684$ & $75\%\pm4\%$ & $0\farcs{}37\pm0\farcs{}02$ & $1.4\pm0.2$ & $21.18\substack{+0.06\\-0.06}$ & $1.452$ & $38\%\pm1\%$ & $22.73\substack{+0.02\\-0.02}$ & $11.13\substack{+0.03\\-0.03}$ &$10.86\substack{+0.10\\-0.10}$\\[3pt]
CID454 & $2.203$ & $36\%\pm3\%$ & $0\farcs{}39\pm0\farcs{}02$ & $0.6\pm0.1$ & $21.20\substack{+0.08\\-0.07}$ & $1.291$ & $9\%\pm1\%$ & $23.35\substack{+0.06\\-0.06}$ & $11.11\substack{+0.04\\-0.04}$ &$10.84\substack{+0.10\\-0.10}$\\[3pt]
CID50 & $5.576$ & $17\%\pm9\%$ & $0\farcs{}16\pm0\farcs{}11$ & $3.2\pm2.2$ & $20.93\substack{+0.86\\-0.48}$ & $4.940$ & $5\%\pm3\%$ & $22.50\substack{+1.15\\-0.55}$ & $11.07\substack{+0.35\\-0.19}$ &$10.80\substack{+0.36\\-0.21}$\\[3pt]
CID543 & $1.902$ & $31\%\pm10\%$ & $0\farcs{}10\pm0\farcs{}00$ & $0.5\pm0.3$ & $21.99\substack{+0.41\\-0.30}$ & $1.435$ & $5\%\pm2\%$ & $23.77\substack{+0.53\\-0.36}$ & $10.70\substack{+0.16\\-0.12}$ &$10.43\substack{+0.19\\-0.15}$\\[3pt]
CID597 & $1.565$ & $42\%\pm17\%$ & $0\farcs{}17\pm0\farcs{}06$ & $1.8\pm0.8$ & $21.87\substack{+0.54\\-0.36}$ & $1.254$ & $12\%\pm1\%$ & $23.56\substack{+0.13\\-0.11}$ & $10.73\substack{+0.22\\-0.15}$ &$10.46\substack{+0.24\\-0.18}$\\[3pt]
CID607 & $1.692$ & $44\%\pm18\%$ & $0\farcs{}21\pm0\farcs{}09$ & $3.4\pm1.1$ & $21.19\substack{+0.58\\-0.37}$ & $2.590$ & $5\%\pm2\%$ & $23.57\substack{+0.51\\-0.35}$ & $11.02\substack{+0.23\\-0.15}$ &$10.75\substack{+0.25\\-0.18}$\\[3pt]
CID70 & $2.041$ & $20\%\pm5\%$ & $0\farcs{}42\pm0\farcs{}10$ & $3.6\pm1.0$ & $21.86\substack{+0.30\\-0.24}$ & $2.361$ & $2\%\pm1\%$ & $24.63\substack{+0.68\\-0.41}$ & $10.98\substack{+0.12\\-0.10}$ &$10.72\substack{+0.16\\-0.14}$\\[3pt]
LID1273 & $1.697$ & $53\%\pm9\%$ & $0\farcs{}30\pm0\farcs{}04$ & $1.2\pm0.5$ & $20.94\substack{+0.21\\-0.18}$ & $2.137$ & $6\%\pm1\%$ & $23.29\substack{+0.15\\-0.13}$ & $11.32\substack{+0.09\\-0.07}$ &$11.05\substack{+0.13\\-0.12}$\\[3pt]
LID1538 & $2.362$ & $44\%\pm8\%$ & $0\farcs{}18\pm0\farcs{}04$ & $2.8\pm0.5$ & $21.25\substack{+0.22\\-0.18}$ & $2.173$ & $8\%\pm1\%$ & $23.09\substack{+0.19\\-0.16}$ & $11.12\substack{+0.09\\-0.08}$ &$10.86\substack{+0.13\\-0.12}$\\[3pt]
LID360 & $3.918$ & $18\%\pm2\%$ & $0\farcs{}63\pm0\farcs{}02$ & $0.8\pm0.4$ & $21.46\substack{+0.14\\-0.12}$ & $4.914$ & $4\%\pm1\%$ & $23.25\substack{+0.17\\-0.15}$ & $11.08\substack{+0.06\\-0.05}$ &$10.81\substack{+0.11\\-0.11}$\\[3pt]
XID2138 & $1.597$ & $39\%\pm6\%$ & $0\farcs{}50\pm0\farcs{}03$ & $1.2\pm0.4$ & $21.87\substack{+0.17\\-0.15}$ & $2.731$ & $5\%\pm1\%$ & $23.90\substack{+0.31\\-0.24}$ & $10.89\substack{+0.07\\-0.06}$ &$10.63\substack{+0.12\\-0.12}$\\[3pt]
XID2202 & $3.23$ & $33\%\pm8\%$ & $0\farcs{}10\pm0\farcs{}00$ & $4.0\pm1.0$ & $21.16\substack{+0.30\\-0.24}$ & $3.852$ & $8\%\pm2\%$ & $22.59\substack{+0.29\\-0.23}$ & $11.15\substack{+0.12\\-0.10}$ &$10.88\substack{+0.16\\-0.14}$\\[3pt]
XID2396 & $3.669$ & $24\%\pm11\%$ & $0\farcs{}58\pm0\farcs{}09$ & $0.8\pm1.4$ & $21.40\substack{+0.65\\-0.40}$ & $5.346$ & $2\%\pm1\%$ & $23.36\substack{+0.24\\-0.20}$ & $11.12\substack{+0.26\\-0.16}$ &$10.85\substack{+0.28\\-0.19}$\\[3pt]
CDFS-1 & $1.358$ & $65\%\pm20\%$ & $0\farcs{}14\pm0\farcs{}07$ & $4.8\pm1.1$ & $22.47\substack{+0.40\\-0.29}$ & ... & ... & ... & $10.71\substack{+0.16\\-0.12}$ &$10.45\substack{+0.19\\-0.15}$\\[3pt]
CDFS-229 & $4.329$ & $18\%\pm2\%$ & $0\farcs{}51\pm0\farcs{}03$ & $0.5\pm0.2$ & $21.57\substack{+0.14\\-0.13}$ & ... & ... & ... & $10.90\substack{+0.06\\-0.05}$ &$10.63\substack{+0.12\\-0.11}$\\[3pt]
CDFS-321 & $3.998$ & $25\%\pm12\%$ & $0\farcs{}38\pm0\farcs{}12$ & $2.3\pm2.0$ & $20.34\substack{+0.70\\-0.42}$ & ... & ... & ... & $11.52\substack{+0.28\\-0.17}$ &$11.25\substack{+0.30\\-0.20}$\\[3pt]
CDFS-724 & $1.355$ & $35\%\pm15\%$ & $0\farcs{}12\pm0\farcs{}03$ & $1.6\pm1.1$ & $23.70\substack{+0.58\\-0.38}$ & ... & ... & ... & $10.06\substack{+0.23\\-0.15}$ &$9.79\substack{+0.25\\-0.18}$\\[3pt]
ECDFS-358 & $2.012$ & $56\%\pm14\%$ & $0\farcs{}36\pm0\farcs{}04$ & $1.7\pm0.5$ & $21.34\substack{+0.30\\-0.24}$ & ... & ... & ... & $11.16\substack{+0.12\\-0.10}$ &$10.89\substack{+0.16\\-0.14}$\\[3pt]
SXDS-X1136 & $1.937$ & $41\%\pm8\%$ & $0\farcs{}10\pm0\farcs{}00$ & $2.0\pm0.5$ & $21.92\substack{+0.23\\-0.19}$ & ... & ... & ... & $10.75\substack{+0.09\\-0.08}$ &$10.49\substack{+0.14\\-0.13}$\\[3pt]
SXDS-X50 & $1.423$ & $41\%\pm9\%$ & $0\farcs{}19\pm0\farcs{}04$ & $1.7\pm0.6$ & $21.99\substack{+0.27\\-0.21}$ & ... & ... & ... & $10.80\substack{+0.11\\-0.09}$ &$10.54\substack{+0.15\\-0.13}$\\[3pt]
SXDS-X717 & $1.426$ & $61\%\pm9\%$ & $0\farcs{}26\pm0\farcs{}07$ & $5.6\pm1.4$ & $21.76\substack{+0.18\\-0.15}$ & ... & ... & ... & $10.77\substack{+0.07\\-0.06}$ &$10.51\substack{+0.12\\-0.12}$\\[3pt]
SXDS-X735 & $2.203$ & $32\%\pm9\%$ & $0\farcs{}22\pm0\farcs{}06$ & $2.0\pm1.0$ & $20.92\substack{+0.33\\-0.25}$ & ... & ... & ... & $11.19\substack{+0.13\\-0.10}$ &$10.92\substack{+0.17\\-0.14}$\\[3pt]
SXDS-X763 & $2.376$ & $6\%\pm4\%$ & $0\farcs{}69\pm0\farcs{}53$ & $2.4\pm0.8$ & $24.13\substack{+1.17\\-0.55}$ & ... & ... & ... & $9.95\substack{+0.47\\-0.22}$ &$9.68\substack{+0.48\\-0.24}$\\[3pt]
SXDS-X969 & $1.613$ & $29\%\pm11\%$ & $0\farcs{}11\pm0\farcs{}02$ & $2.1\pm1.1$ & $21.59\substack{+0.52\\-0.35}$ & ... & ... & ... & $11.03\substack{+0.21\\-0.14}$ &$10.76\substack{+0.23\\-0.17}$\\[3pt]
\enddata
\label{tab:result_sersic}
\tablecomments{
Column~1: Object ID.
Column~2-6: WFC3 inference. 
Column~2: Reduced $\chi ^2$ value by the best PSF in the library.
Column~7-9: ACS inference. 
Column~10: Observed host luminosity in the rest-frame R band.
Column~11: Host total stellar mass.
}
\end{deluxetable*}

\section{Results}
\label{sec:result}

\subsection{Host galaxy properties}
We derive the rest-frame R band magnitude ($M_R$) and luminosity ($L_R$) of our sample based on their filter magnitude. 18/32 systems have multi-band host magnitude which enables us to select the best stellar population that could fit the overall sample. We find that the 1~Gyrs stellar population with solar metallicity by Chabrier initial mass function (IMF)~\citep{Bruzual2003} could well match the color between the WFC3 and ACS band magnitude of these samples.

Adopting the color information, we perform the K-correction and derive the rest-frame R band magnitude of our sample. Given that the WFC3 filter we selected is very close to the rest-frame R band, we expect the $M_R$ uncertainty introduced by this K-correction is within 0.05~mag. We derive the rest-frame R band luminosity by $\log L_R/L_{R, \odot} = 0.4 (M_{R, \odot}-M_R)$, where $M_{R, \odot}=4.61$~\citep{Blanton07}.

Taking the stellar population, we estimate the stellar mass content of each host galaxy according to the mass-to-light ratio of the stellar population we adopted. The uncertainty level associated with the stellar mass is expected to be of order 0.1~dex, at fixed stellar initial mass function (changing the initial mass function would change all our stellar masses systematically).


The inferred $L_R$ and $M_*$ are listed in Table~\ref{tab:result_sersic}. In the rest of this section, we compare their relations to the \mbh.

\subsection{\mbh-\lhost\ relation}\label{sec:ml}
We show our measurements of the \mbh-\lhost\ relation together with the comparison sample in Figure~\ref{fig:ML}. Following common practice, we fit the local relation as a linear relation,
\begin{eqnarray}
\label{eq:MLlocal}
\log \big( \frac{\mathcal M_{\rm BH}}{10^{7}M_{\odot}})= \alpha + \beta \log(\frac{L_R}{10^{10}L_{\odot}}),
\end {eqnarray}
which enables direct comparison to the distant sample. The inferred local relations, as presented in the Figure~\ref{fig:ML}, shows that the observed \mbh-\lhost\ relation for our measurements is nearly identical to the local and intermediate ones. The observational result indicates that the relation between black hole mass and the host luminosity are similar at different period of the universe.

Considering that both the black hole and host galaxy are expected to evolve over cosmic time, it is essential to transfer high-$z$ measurements to their values at today so as to compare them in an equivalent frame. We will make this comparison taking the evolution effect into account in Section~\ref{sec:ml-ev}.

\begin{figure}
\centering
{\includegraphics[width=0.5\textwidth]{fig/MBH-L_obs.pdf}}
\caption{\label{fig:ML} 
Illustration of observed correlations between \mbh-\lhost, with local linear relation indicated. 
%The redshift are color-coded, for distant and intermediate AGNs.
The black line and the blue colored equation indicates the best-fit result of local sample as Equation~\ref{eq:MLlocal}, with $1-\sigma$ confidence interval colored as gray region.
}
\end{figure} 

\subsection{\mbh-\smass\ relation}\label{sec:mm}
In Figure~\ref{fig:MM}, we plot the \mbh-\smass\ relation and compare to the local and intermediate sample. We visually see that the relations for the three samples are close from one to another. Similar as Equation~\ref{eq:MLlocal}, we consider the local \mbh-\smass\ relations as linear relation as:
\begin{eqnarray}
\label{eq:MMlocal}
\log \big( \frac{\mathcal M_{\rm BH}}{10^{7}M_{\odot}})= \alpha + \beta \log(\frac{M_*}{10^{10}M_{\odot}}).
\end {eqnarray}

To understand the evolution for high-$z$ relations, we fit the offset as a function of redshift in form as:
\begin{eqnarray}
\label{eq:offset}
\Delta \log \mathcal M_{\rm BH}= \gamma \log (1 + z),
\end{eqnarray} 
where $\Delta \log \mathcal M_{\rm BH} = \log \big( \frac{\mathcal M_{\rm  BH}}{10^{7}M_{\odot}}) -\alpha-\beta\log(\frac{M_*}{10^{10}M_{\odot}})$. We obtain $\gamma  = 0.74 \pm 0.21$, i.e. tentative evidence for evolution, see Figure~\ref{fig:MM-vz}, panel (a). We also show the ratio of \mbh-\smass\ of the three samples at different redshift in Figure~\ref{fig:MM-vz}, panel (b). 

\begin{figure}
\centering
{
\includegraphics[width=0.5\textwidth]{fig/MBH-Mstar.pdf}
}
\caption{\label{fig:MM} 
Similar to the Figure~\ref{fig:ML}, but the comparison of the \mbh-\smass\ relations.}
\end{figure} 

\begin{figure*}
\centering
\begin{tabular}{c c}
{\includegraphics[width=0.5\textwidth]{fig/MBH-Mstar-vz_style1.pdf}}&
{\includegraphics[width=0.5\textwidth]{fig/MBH-Mstar-vz_style0.pdf}}\\
\end{tabular}
\caption{\label{fig:MM-vz} 
Left: illustration of the offset in log\mbh\ (VS. \smass) as a function of redshift. The orange band is the intrinsic scatter of local linear relation. Right: \mbh/\smass\ as a function of redshift. 
{\color{blue} We use green open circles to show the expected shifting of the median value of the $\Delta$\mbh\ before and after consider the selection effect using framework by \citet{Schulze2011} in Section~\ref{sec:sf_framework}.}
{\color{red} [Xuheng: The blue points are from Cisternas et al. 2011, they are consistent to the local ones.]}
}
\end{figure*} 

We visually see that in Figure~\ref{fig:MM-vz}, panel (a), most of our measurements locate within the scattering range of the local sample. We doubt whether the positivity of $\gamma$ stems from the three outliers (i.e., the top-three systems in Figure~\ref{fig:MM}). We thus exclude these systems and re-run the statistical analysis to see if offset still exists among the rest of our sample. As a result, we obtain that $\gamma  = 0.55 \pm 0.29$, indicating the positive evolution trend become weaker but still exists, as shown in Figure~\ref{fig:MM_restsample}.

\begin{figure}
\centering
{\includegraphics[width=0.5\textwidth]{fig/MBH-Mstar-vz_subsample.pdf}}
\caption{\label{fig:MM_restsample} 
Same as Figure~\ref{fig:MM-vz}, left, but excluding the 3 outliers systems. The result shows the evolution still exist among the other 29 AGN sample.
}
\end{figure} 

It is worth note that the local sample employed are bulge-dominated galaxies and the adopted \smass\ are equivalent to their bulge masses. That is, we are comparing the \mbh-\smass$_{,total}$ at the distant universe to the \mbh-\smass$_{,bulge}$ at today. Considering that a major part of the distant systems has indicated the disk component (i.e., the systems with fitted \sersic\ index close to 1), the \smass$_{,bluge}$ of them are supposed to be even less.

Without taking into account the selection effect, the direct comparison shows that the growth of the black hole predates that of the host bulge.

\subsection{Selection effect}
\label{select_eff}

Our AGN sample is selected based on the value of \mbh, Eddington ratio, and \halpha\ broad emission line, with low nuclear-to-host ratios. The inference could be biased, when extending the selected sample to the global without properly considering the selection effect~\citep{Tre++07, Bennert++2011, Schulze2014, Park15}. In theory, the simulations~\citep{DeG++15} also find the scaling relations by the selected sample has steeper feature than the random ones, highlighting the necessity of eliminating the selection biases. {\color{blue} We consider the selection effect using two independent approach introduced in the following two subsections.

\subsubsection{Test selection effect using Monte Carlo simulation}
We take selection effects into account using the Monte Carlo simulation based on the approach introduced by \citet{Tre++07}. Following \citet{Park15, Ding2017b}, we generate the simulating samples from a combination of the local active \mbh\ function by \cite{Schulze2010} and the local samples adopted in this work. We then added Gaussian random noise to the simulated samples as a function of two free parameters, $\gamma$ and \sint, where \sint\ is the intrinsic scatter when considering the scaling relation as linear. According to the observed distribution, we consider the observational selection of \mbh\ of our samples by their lower and upper limits to model their distribution. To evaluate the posterior at the given $\gamma$ and \sint, for each object, we consider the likelihood of observing such object using the simulated sample, and consider whether the object would be selected or not, given the sensitivity. In addition, we adopt both uniform (flat) prior and lognormal prior for \sint\ (using the inference from linear fitting).

Combining the 32 AGNs together with the intermediate redshift sample, we present the inferred  $\gamma$ and \sint\ in the two-dimensional planes in Figure~\ref{fig:select_effect}. With a lognormal prior for \sint, the fitted $\gamma = 0.8\pm0.3$, consistent with the inference by the flat prior. These inference are consistent with the previous values without considering the selection effect in Section~\ref{sec:mm}, indicating that the evolution feature of our sample exist globally in the overall samples. Furthermore, we find that the final inferred scatter \sint$\sim 0.3$ are similar to the local scatter, suggesting that the scatter of the scaling relation at high redshift are similar to local ones and indicating that we have not significantly underestimated/overestimated our errors for high redshift samples.

We also study the selection effect by only consider the new 32 AGNs, result in a higher evolution trend with a higher value of $\gamma$. We show all the result of the $\gamma$ and \sint\ in Table~\ref{table:gamma_sf}.

\begin{table}
\centering
    \caption{The summary for the different inference of $\gamma$.}\label{table:gamma_sf}
     \resizebox{8cm}{!}{
     \begin{tabular}{cccc}
     \hline
     Sample & Selection effects &  \mbh-\smass & \mbh-\lhost \\
     &&&\\
     \hline\hline
32 AGNs + intermediate & No & 0.74$\pm$0.21 & 0.70$\pm$0.18 \\
32 AGNs + intermediate& Yes & 0.80 $\pm$ 0.30 & 0.60 $\pm$ 0.30 \\
32 AGNs & No & 0.80$\pm$0.27  & 1.13$\pm$0.23\\
32 AGNs& Yes & 1.30$\pm$0.50 & 1.70$\pm$0.40 \\
     \hline
     \end{tabular}}
    \tablecomments{
    Entire sample includes the 32 AGNs and the intermediate redshift AGNs from the reference in Section~\ref{sec:compare_sample}. Note that the adopted \lhost\ have been transferred to today assuming the passive evolution scenario using Equation~\ref{eq:L_relation}.
The results of selection effects used the lognormal prior of \sint.
}
\end{table}

\begin{figure*}
\centering
\begin{tabular}{c c}
\subfloat[\mbh-\smass, flat prior]
{\includegraphics[width=0.5\textwidth]{fig/MM_MC_seleff_flatprior.pdf}}&
\subfloat[\mbh-\smass, lognormal prior]
{\includegraphics[width=0.5\textwidth]{fig/MM_MC_seleff_lognormprior.pdf}}\\
\end{tabular}
\caption{\label{fig:select_effect} 
Constraining the evolution factor $\gamma$ of \mbh-\smass\ relation using Equation~\ref{eq:offset}, with intrinsic scatter \sint, using Monte Carlo simulation, by taking selection effects into account. The adopted sample includes 32 AGNs and the intermediate redshift AGNs, using flat prior of \sint\ (left panel) and lognormal prior (right panel).
}
\end{figure*} 

\subsubsection{Test selection effect using Bayesian framework} \label{sec:sf_framework}
We use the Bayesian framework introduced by \citet{Schulze2011} to estimate the expected bias that could be raised by our selection window and local sample that adopted. Given the particular selection window (\mbh,  $L_{\rm bol}$ and Eddington ratio $\lambda$), the framework considers the fact that the intrinsic scatter in the BH-bulge relation has trend to decline regime of the stellar mass function, result in an average stellar masses lower than the relation. Moreover, the other relevant sources of bias are considered in this framework, for example the active fraction among SMBHs varies by the black hole mass function, in the sense that BHs with higher mass are less likely to be active than lower mass ones. 

Considering our specific selection limits: $\log($\mbh$)~\in[7.5, 8.56]$, $\log(L_{\rm bol}) \in [45.0, 46.2] $ and  $\log(\lambda) \in [-2.0, 0.5]$, we infer an expected bias of 0.21~dex in the observed $\Delta\log($\mbh$)$ for samples at $z\sim1.5$. Interestingly, this bias could account for the offset that obtained in the \mbh-\smass\ correlations in Section~\ref{sec:mm}. We use green open circles to point out the shifting of the expected population given these biases, as shown in Figure~\ref{fig:MM-vz}.

Our conclusion is that the considering the selection effect using framework by \citet{Schulze2011}, we get consistent \mbh-\smass\ with the local sample.

}

\section{discussion}
\label{sec:dis}

\subsection{Host luminosity passive evolution correction}\label{sec:ml-ev}
In the passive evolution scenario, we expect the galaxy luminosity to fade over time. Thus, we transfer the \lhost\ for distant samples at today so as to compare the \mbh-\lhost\ relation to the local in the equivalent frame.
We consider this scenario following \citet{Ding2017b} by parametrizing the luminosity evolution with the functional form as
$d{\rm mag}_{\rm R}\sim~d\log(1+z)$, i.e.,
\begin{eqnarray}
\label{eq:L_relation}
\log(L_{R,0})=\log(L_{R}) - 1.48 \log (1+z).
\end{eqnarray} 
This formalism is more accurate to fit a broad range redshift comparing to a single slope as $d$mag$/dz$. We refer the interested readers to \citet[][section 5.4]{Ding2017b} for more details.

Having transferred the \lhost\ to today, we find that, as showing in Figure~\ref{fig:ML-vz}-(a), at fixed mass, the BH in the more distant universe tends to reside in less luminous hosts, which is consistent to the \mbh-\smass\ relation. We fit the offset as a function of redshift in form as Equation~\ref{eq:offset} and obtain $\gamma = 0.71 \pm 0.18$, as showing in Figure~\ref{fig:ML-vz}-(b). We further consider the selection effect using the previous approach as introduced in Section~\ref{select_eff}, and obtain $\gamma = 0.5\pm0.5$ and $\gamma = 0.6\pm0.3$, with flat and lognormal prior, respectively, as showing in Figure~\ref{fig:ML-vz}-(c), (d).

The inferred $\gamma$ here could have large systematic given the limitations in this section. First, the passive evolution is based on a simplified correction; after all, we are not clear exactly how the host evolves to $z=0$. Moreover, we only consider the evolution of the host galaxy and assume the \mbh\ does not change much. %Moreover, in the literature, the correlation of the \mbh-\smass\ is considered as more fundamental which would be more efficient to understand their co-evolution.

\begin{figure*}
\centering
\begin{tabular}{c c}
\subfloat[\mbh-\lhost\ relation, evolution-corrected]
{\includegraphics[height=0.5\textwidth]{fig/MBH-L_ev.pdf}}&
\subfloat[offset in log\mbh\ (VS. \lhost) as a function of redshift]
{\includegraphics[height=0.5\textwidth]{fig/MBH-L-vz.pdf}}\\
\subfloat[\mbh-\lhost, flat prior]
{\includegraphics[width=0.5\textwidth]{fig/ML_MC_seleff_flatprior.pdf}}&
\subfloat[\mbh-\lhost, lognormal prior]
{\includegraphics[width=0.5\textwidth]{fig/ML_MC_seleff_lognormprior.pdf}}\\
\end{tabular}
\caption{\label{fig:ML-vz} 
Same as previous figures but for \mbh-\lhost relation, considering the passive evolution correction for host galaxy luminosity.}
\end{figure*} 

{\color{blue}
\subsection{The origin of the offset}
{\color{red}[Xuheng:  Given their larger uncertainty, do we still want to present these figures?]}
We seek for the cause that could raise the evolution in the BH and host galaxy correlations at high redshift. We consider if the offset is related to the AGN activities. The straightforward way is to study such offset, i.e. $\Delta \log \mathcal M_{\rm BH}$, as a function of \mbh\ and Eddington ratios, showing in Figure~\ref{fig:offset_vs_MBH_edd}. Interestingly, we visually see the trend that the offset turns to be larger with at value of \mbh. However, at higher Eddington ratio, the offset turns to be lower.
 {\color{red} Some interpretations need to be presented.}

\begin{figure*}
\centering
\begin{tabular}{c c}
\subfloat[Offset of the \mbh-\smass\ relation as a function of \mbh.]
{\includegraphics[height=0.5\textwidth]{fig/offset_vs_MBH.pdf}}&
\subfloat[Offset of the \mbh-\smass\ relation as a function of Eddington ratio.]
{\includegraphics[height=0.5\textwidth]{fig/offset_vs_Edd.pdf}}\\
\end{tabular}
\caption{\label{fig:offset_vs_MBH_edd} 
Illustration of the offset for high-$z$ \mbh-\smass\ sample as a function of \mbh\ (left) and Eddington ratio (right).}
\end{figure*} 
}

\subsection{Systematic error}
In this work, we use the start-of-the-art techniques to derive the host flux from the AGN image. The fidelity of the inferred apparent magnitudes of the host galaxies are high. Nevertheless, we adopt one stellar population to the overall sample to derive the rest-frame R band luminosity and stellar mass.
We note that adopting the private stellar populations for each samples which havs both multi-band host magnitude could be more robust in principle. However, considering that the host magnitude in ACS band is faint (host-total flux ratio $< 30\%$), the fidelity of the SED fitting is actually lost. Perhaps the best way to derive the stellar population is to combining our measurements to the ground-based photometry.

We highlight that, in this particular paper, we focus most on the decomposing of the AGN image to present our inference of the host. In the mean time, we carry out a rapid analysis of the \mbh\ and host properties relations. The measurements incorporating the ground-based photometry and the comparison with the simulations are presented in our companion paper.

At last, we stress that the uncertainty in the scaling relations is dominated by the single epoch black hole mass estimates. That is to say, the intrinsic scatter is not necessarily worse at high-$z$ than the low-$z$.

\subsection{Systematic with different local anchor}
To compare our high-$z$ sample to the local relation, we adopt the local measurements by \citet{Ben++10, Bennert++2011, H+R04}. In the literature, there are other local sample have been measured which also usually be considered as the local anchor. We note that different local samples have inconsistency which could change the result of the co-evolutions. 

To estimate this effect, we compare our local \mbh-\smass\ sample to the local quiescent galaxies by \citet{Kormendy13} (hereafter K13). We find that K13's local \mbh-\smass\ relation would raise the anchor by $\sim$0.3~dex and totally erase the offset of $\Delta\log$\mbh. However, we also note that the \mbh-\sigstar\ relations in K13 is also higher than the one by our local sample. Note that the \mbh-\sigstar\ relations by the quiescent galaxy are usually used to determine the virial coefficient for AGNs. That is, adopting the K13 would boost the value of our \mbh\ by $\sim$0.3~dex as well. As a result, our inference of the fitted offset $\gamma$ still remains at the same level.

This quick comparison of the local sample highlights the need of the solid local baseline for the future studies (Bennert et al., in prep).

%\begin{figure}
%\centering
%{\includegraphics[width=0.5\textwidth]{fig/local_MM_comparison.pdf}}
%\caption{\label{fig:local_MM_comp} 
%The local \mbh-\smass\ relations from different literature. The black and red line are the best-fit linear relation using the sample by combining B11+HR04 and B11+ K13, respectively, showing a 0.3~dex offset in the intercept value.
%}
%\end{figure} 

\section{Summary} \label{sec:sum}
We studied the evolution of the correlations between the supermassive black hole and their host galaxies using new measurements of 32 X-ray selected AGNs at $1.2<z<1.7$.

We used the published near-infrared spectroscopic observations \halpha\ and \hbeta\ emission lines to estimate the reliable values of BH mass. To obtain the properties of the AGN host galaxies, we performed the AGN image decomposition using the state-of-the-art techniques. We adopted the \hst/WFC3 infrared channel to observe the high-resolution AGN imaging data and collected the PSF-stars in across all the fields to build to a library of PSF for the fitting. Using the latest image modeling tool \lenstronomy, we decomposed the AGN image in 2-D plane taking each PSF in the library. We obtained the host \sersic\ property (i.e., host flux, effective radius, \sersic\ index) using a weighted arithmetic mean based on the inference by the eight top-ranked PSFs. Incorporating with the inference by \hst/ACS image, we selected the stellar population, hence derived the rest-frame R band Luminosity (\lhost) and stellar mass of the host (\smass).

Combining our high-$z$ measurements with the intermediate redshift samples from the literature~\citep{Park15, Bennert11, SS13, Cisternas2011}, we compared the scaling relations to the ones by the local with main results summarized as follows:
\begin{enumerate}
\item The \mbh-\smass\ relations at higher redshift are inconsistent with the local ones, i.e. Figure~\ref{fig:MM}, \ref{fig:MM-vz}.
\item Considering the correction for passive evolution of galaxy luminosity, the \mbh-\lhost\ relations at higher redshift are inconsistent with the local ones,  i.e. Figure~\ref{fig:ML-vz}.
\item The offsets of by the evolution are well described by the form $\Delta\log\mathcal M_{\rm BH}= \gamma \log (1 + z)$. Taking intro account selection effect, we obtain $\gamma = 0.8\pm0.3$ (\mbh-\smass) and $\gamma = 0.6\pm0.3$ (\mbh-\lhost).
\item We compare the properties of our active galaxy sample to the ones from inactive galaxy in the CANDLES imaging survey at the same redshift range, finding no significant discrepancy among them.
\end{enumerate}

Our inference indicates the evidence of disk+bluge component, i.e. the ones with \sersic\ $n<2$, and the local comparing samples are bulge dominated sample, we hence conclude that the BHs 8-10~Gyrs ago reside in bulges, if not entire galaxies, which are less massive/luminous than today.

Towards a better estimation of the evolution of the scaling relations, we need to address the limitations of this work. First, most of the host-total ration in the \hst/ACS band, the fidelity of the host flux are thus limited at UV. We thus adopt a universal stellar template to the overall galaxies based on the general inference between UV band and IR band, rather than carry out the SED fitting for each galaxy, to derive the \lhost\ and \smass. Second, at this moment, the local scaling relations that we current have are not yet solid. We have found the values of $\gamma$ could vary from 0.8 to 2.0, using different local sample from different literature \citep{Ben++10, Bennert++2011, H+R04, Kormendy13, Bentz2018}. Thus, a solid sample served as the local anchor still needed to be well defined (Bennert et al., in prep). One the other hand, comparing the observational measurements to the ones by the simulation could avoid not only the dependency to the local anchor, but also the bias by selection effect \citep{DeG++15, Khandai2015, Menci2014}. To overcome these limitation, we are currently incorporating our measurement with the ground-based photometry to derive SED fitting, in order to compare with the sample by simulations, which will be presented in our coming up paper.

Looking into the future, the forthcoming lunching of {\it James Webb Space Telescope} ({\it JWST}) would provide the high quality imaging data of AGN at higher redshift (up to $z\sim$7-10) and thus trace the evolution of correlation at more distant universe. Furthermore, {\it JWST} is capable of showing the morphology of structures feeding black hole accretion and feedback on the host galaxy~\citep{Ford2014}, thus providing the opportunity to understand the origin of the scaling relations and their evolution.


\section*{Acknowledgments}
Based in part on observations made with the NASA/ESA Hubble Space Telescope, obtained at the Space Telescope Science Institute, which is operated by the Association of Universities for Research in Astronomy, Inc., under NASA contract NAS 5-26555. These observations are associated with programs
\#15115. Support for this work was provided by NASA through grant number HST-GO-15115 from the Space Telescope Science Institute, which is operated by AURA, Inc., under NASA contract NAS 5-26555.

The authors fully appreciate the discussions with Hyewon Suh, Vardha N. Bennert, Anton M. Koekemoer, Takahiro Morishita and Peter Williams.

XD, SB and TT acknowledge support by the Packard Foundation through a Packard Research fellowship to TT.

This work has made use of \lenstronomy~\citep{lenstronomy}, {\sc Astropy}~\citep{Astropy}, {\sc photutils}~\citep{photutils}, {\sc TOPCAT}~\citep{TOPCAT}, {\sc Matplotlib}~\citep{Matplotlib} %, {\sc adjustText}~\citep{adjustText}
and standard Python libraries.

\bibliographystyle{apj.bst}
\bibliography{references}

\newpage
\appendix

\section{A. Comparison to the inactive galaxies}\label{sec:comp_inactive}
We aim to compared the morphology of AGNs host galaxies to the inactive ones. From the CANDLES imaging survey. we collect 4401 inactive galaxies in a comparable redshift range ($1.2<z<1.7$) whose \sersic\ properties are inferred by \citet{VDwel++2012} by \galfit, and their stellar masses are derived by 3-D-\hst\ spectroscopic survey~\citep{Momcheva2016, Brammer2012}.
%In Figure~\ref{fig:Mstar-rn}, we plot the \smass\ versus the \Reff\ and \sersic\ index. The color coding is based on the filter flux ratio between WFC3 and ACS. The distribution of \smass-\Reff\ relation show that our AGN hosts concentrate at the high end of the stellar mass, close to the red sequence. 
%\begin{figure*}
%\centering
%\begin{tabular}{c c}
% \hspace{-2.5em}
%{\includegraphics[trim = 0mm 0mm 90mm 0mm, clip, height=0.45\textwidth]{fig/Mstar-Reff.pdf}}&
%{\includegraphics[ height=0.45\textwidth]{fig/Mstar-Sn.pdf}}\\
%\end{tabular}
%\caption{\label{fig:Mstar-rn} 
%The comparison of the galaxy properties between our 32 AGNs' hosts to the CANDLES sample including 4401 inactive galaxies (circles). The color coding is based on the filter flux ratio between WFC3 and ACS. For CANDLES sample, the WFC3/F125W flux is taken. The black squares are the AGN samples with only WFC3 band observation.
%}
%\end{figure*} 
We compare the histogram of the inferred \Reff\ and \sersic\ index to the inactive galaxies in Figure~\ref{fig:hist_rn}, where there is no significant difference between their distribution and median value. We also compare the difference by performing the Kolmogorov-Smirnov test and inferring the p-value as 0.42 and 0.04 for \Reff\ and $n$, respectively.

We conclude that the host galaxies of our AGN sample are representative of the overall population of galaxies at comparable luminosity and stellar mass at the same redshift.
\begin{figure*}[ht]
\centering
\begin{tabular}{c c}
{\includegraphics[ height=0.3\textwidth]{fig/Hist_Reff.pdf}}&
{\includegraphics[ height=0.3\textwidth]{fig/Hist_Sn.pdf}}
\end{tabular}
\caption{\label{fig:hist_rn} 
The comparison of the histogram of the \Reff\ and \sersic\ n, with median value indicated.}
\end{figure*} 

\newpage
\section{B. The demonstration of remaining 31 AGNs}\label{sec:restsample}
The AGNs decomposition for the other 31 objects of our sample, presented in the same way as Figure~\ref{fig:AGN_decomp}.

\begin{figure}[ht]
\centering
%\hspace{-5.5em}
{
\includegraphics[height=0.25\textwidth]{fig/best_fit_CID1281_SB_profile.pdf}
\includegraphics[height=0.25\textwidth]{fig/best_fit_CID206_SB_profile.pdf}
\includegraphics[height=0.25\textwidth]{fig/best_fit_CID216_SB_profile.pdf}
\includegraphics[height=0.25\textwidth]{fig/best_fit_CID237_SB_profile.pdf}
}
\end{figure} 

\begin{figure}
\centering
%\hspace{-5.5em}
{
\includegraphics[height=0.25\textwidth]{fig/best_fit_CID255_SB_profile.pdf}
\includegraphics[height=0.25\textwidth]{fig/best_fit_CID3242_SB_profile.pdf}
\includegraphics[height=0.25\textwidth]{fig/best_fit_CID3570_SB_profile.pdf}
\includegraphics[height=0.25\textwidth]{fig/best_fit_CID452_SB_profile.pdf}
\includegraphics[height=0.25\textwidth]{fig/best_fit_CID454_SB_profile.pdf}
}
%\figurenum{1}
%\caption{Continued.}
\end{figure} 

\begin{figure*}
\centering
%\hspace{-5.5em}
{
\includegraphics[height=0.25\textwidth]{fig/best_fit_CID50_SB_profile.pdf}
\includegraphics[height=0.25\textwidth]{fig/best_fit_CID543_SB_profile.pdf}
\includegraphics[height=0.25\textwidth]{fig/best_fit_CID597_SB_profile.pdf}
\includegraphics[height=0.25\textwidth]{fig/best_fit_CID607_SB_profile.pdf}
\includegraphics[height=0.25\textwidth]{fig/best_fit_CID70_SB_profile.pdf}
}
%\figurenum{1}
%\caption{Continued.}
\end{figure*} 

\begin{figure*}
\centering
%\hspace{-5.5em}
{
\includegraphics[height=0.25\textwidth]{fig/best_fit_XID2138_SB_profile.pdf}
\includegraphics[height=0.25\textwidth]{fig/best_fit_XID2202_SB_profile.pdf}
\includegraphics[height=0.25\textwidth]{fig/best_fit_XID2396_SB_profile.pdf}
\includegraphics[height=0.25\textwidth]{fig/best_fit_LID1273_SB_profile.pdf}
\includegraphics[height=0.25\textwidth]{fig/best_fit_LID1538_SB_profile.pdf}
}
%\figurenum{1}
%\caption{Continued.}
\end{figure*} 

\begin{figure*}
\centering
%\hspace{-5.5em}
{
\includegraphics[height=0.25\textwidth]{fig/best_fit_LID360_SB_profile.pdf}
\includegraphics[height=0.25\textwidth]{fig/best_fit_CDFS-1_SB_profile.pdf}
\includegraphics[height=0.25\textwidth]{fig/best_fit_CDFS-229_SB_profile.pdf}
\includegraphics[height=0.25\textwidth]{fig/best_fit_CDFS-321_SB_profile.pdf}
\includegraphics[height=0.25\textwidth]{fig/best_fit_CDFS-724_SB_profile.pdf}
}
%\figurenum{1}
%\caption{Continued.}
\end{figure*} 

\begin{figure*}
\centering
%\hspace{-5.5em}
{
\includegraphics[height=0.25\textwidth]{fig/best_fit_ECDFS-358_SB_profile.pdf}
\includegraphics[height=0.25\textwidth]{fig/best_fit_SXDS-X1136_SB_profile.pdf}
\includegraphics[height=0.25\textwidth]{fig/best_fit_SXDS-X50_SB_profile.pdf}
\includegraphics[height=0.25\textwidth]{fig/best_fit_SXDS-X717_SB_profile.pdf}
\includegraphics[height=0.25\textwidth]{fig/best_fit_SXDS-X735_SB_profile.pdf}
}
%\figurenum{1}
%\caption{Continued.}
\end{figure*} 

\begin{figure*}
\centering
%\hspace{-5.5em}
{
\includegraphics[height=0.25\textwidth]{fig/best_fit_SXDS-X763_SB_profile.pdf}
\includegraphics[height=0.25\textwidth]{fig/best_fit_SXDS-X969_SB_profile.pdf}
}
%\figurenum{1}
%\caption{Continued.}
\end{figure*} 

\end{document}