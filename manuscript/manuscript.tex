\documentclass[apj]{emulateapj}
\usepackage{times}
%\newcommand{\mdot}{M$_{\odot}$\,}
\usepackage{natbib}

\usepackage[backref,breaklinks,colorlinks,citecolor=cyan]{hyperref} 
%\setlength{\parindent}{0pt}
\usepackage{booktabs}
\usepackage{graphicx}
\usepackage{bm}
%\usepackage{deluxetable}
\usepackage{multirow}
\usepackage{enumitem}
\shorttitle{TDLMC. l}
\shortauthors{X. Ding et al.}

\begin{document}

\def\lcdm{$\Lambda$CDM}
\def\hst{{\it HST}}
\def\efr{$R_{\mathrm{eff}}$}
\def\galfit{\sc Galfit}
\def\mbh{$\mathcal M_{\rm BH}$}
\def\lhost{$L_{\rm host}$}
\def\jcap{Journal of Cosmology and Astroparticle Physics}
\def\halpha{${\it H}\alpha$}
\def\hbeta{${\it H}\beta$}

\title{QSO decomposition}

\author{Xuheng Ding\altaffilmark{1, 2}, Tommaso Treu\altaffilmark{2}, Anowar J. Shajib\altaffilmark{2}, Dandan Xu\altaffilmark{3}, Geoff C.-F. Chen\altaffilmark{4},  Anupreeta More\altaffilmark{5},
}
\email{dxh@astro.ucla.edu}
% Dandan Xu, Giulia, Anupreeta, Chih-Fan, Daniel, Chris, Phil Marshall, Dominique, 

\altaffiltext{1}{School of Physics and Technology, Wuhan University, Wuhan 430072, China}
\altaffiltext{2}{Department of Physics and Astronomy, University of California, Los Angeles, CA, 90095-1547, USA}
\altaffiltext{3}{Heidelberg Institute for Theoretical Studies, Schloss-Wolfsbrunnenweg 35, D-69118 Heidelberg, Germany}
\altaffiltext{4}{Department of Physics, University of California, Davis, CA 95616, USA}
\altaffiltext{5}{Kavli IPMU (WPI), UTIAS, The University of Tokyo, Kashiwa, Chiba 277-8583, Japan}
\altaffiltext{6}{Max Planck Institute for Astrophysics, Karl-Schwarzschild-Strasse 1, D-85740 Garching, Germany}
\altaffiltext{7}{Exzellenzcluster Universe, Boltzmannstr. 2, 85748 Garching, Germany}
\altaffiltext{8}{Ludwig-Maximilians-Universit{\"a}t, Universit{\"a}ts-Sternwarte, Scheinerstr. 1, 81679 M{\"u}nchen, Germany}
\altaffiltext{9}{Kavli Institute for Particle Astrophysics and Cosmology, Stanford University, 452 Lomita Mall, Stanford, CA 94035, USA}
\altaffiltext{10}{STAR Institute, Quartier Agora - All\'ee du six Ao\^ut, 19c B-4000 Li\`ege, Belgium}

\begin{abstract}
To be written. To be written. To be written. To be written. To be written. To be written. To be written. To be written. To be written. To be written. To be written. To be written. To be written. To be written. To be written. To be written. To be written. To be written. To be written. To be written. To be written. To be written. To be written. To be written. To be written. To be written. To be written. To be written. To be written. To be written. To be written. To be written. To be written. 
\end{abstract}

\keywords{galaxies: active --- galaxies: evolution}

\section{Introduction}
\label{sec:introduction}
skeleton
\\
1. The ML relation and why understand its evolution is important.\\
2. To understand this evolution, it is important to trace to high redshift. So, why it is difficult.\\
3. Introduce reference. Woo's. SS13, Treu's, Park79. In the theory, DeGraf...\\
4. This work, we introduces samples till redshift 1.7.

\section{Observations and data reduction}
\label{sec:data}
In this section, we summarize the sample selection, observations, and data reduction of our sample. 

\subsection{Sample selection}
%How the samples are selection.\\
%Where are they from?
%\\The redshift range? In a table?

We aim to study the relation between the BH masses (\mbh) and their host galaxy properties including luminosities (\lhost) stellar mass to redshift at $z>1$. For this propose, we focus on the broad-line (type-1) AGNs as provided by the X-ray coverage of COSMOS \citep{Civano2016}, (E)-CDFS-S \citep{Lehmer2005, Xue2011}, and SXDS \citep{Ueda2008} fields. We select the broad-line AGNs at redshift region $1.2<z<1.7$ which cover a BH range $7.5<{\rm log}$\mbh $<8.5$. The Near IR spectra of AGNs are available from the survey of Subaru's Fiber Multi-Object Spectrograph \citep[FMOS, ][]{Kimura2010, Schulze2018}, a near-infrared (0.9-1.8 $\mu m$) spectrograph have with 400 apertures. The FMOS survey provides the best \mbh\ estimates by \halpha\ and \hbeta\ lines out to $z\sim1.7$ \citep{Greene2005, Matsuoka2013, Nobuta2012}.

Our final sample is composed of 36  AGNs; 32 of them are new systems \textcolor{red}{\bf (32+4?, the information of these four sample?)}. To compare with the samples in the literature, we adopt the samples from ... \textcolor{blue}{\bf The samples from H0licow7}
For the calibration with the local relation, we introduce the well-defined 19 local AGN measurements \citep{Ben++10, Peterson2004} which defines the zero-point. Table ? list all the samples.

\subsection{Observation and reduction}
\label{observation}
Introduce the QSO surface photometry\\
Introduce the AGN broad line Emission

\subsubsection{HST image for photometry}
The QSO surface photometry in IR Channel Filters.\\
The ACS data.
\subsubsection{AGN Emission line}
\label{sec:bh_mass}
The introduction of the BH inferred by the board emission line.

\section{Analysis}
\label{sec:Analysis}

\subsection{QSO image decomposition}
\subsubsection{PSF library}	
\label{sec:psf_library}
Introduce the PSF library

\subsubsection{Modeling method}
Introduce the lenstronomy. Model the PSF in each library.\\
Inspired by Simon's work, we rank the performance of the each PSF and weight for the final fitting result. 

\subsubsection{host luminosity and stellar mass estimates}
k-correction by the galaxy model assumption\\
stellar mass \\

\subsection{BH masses}
Introduce different recipes as used in the paper.

\section{Results}
\label{sec:result}


\section{Summary}
\label{sec:sum}
We presented the study of the Mbh and host property's evolution to high redshift until 1.8.

\section*{Acknowledgments}
We thank Vivien Bonvin, Simon Birrer, Matthew W. Auger, Xiao-Lei Meng for useful suggestions and technical supports. X.D. acknowledges support by China Postdoctoral Science Foundation Funded Project; he is also grateful for Zong-Hong Zhu's support and funding. T.T. acknowledges support by the Packard Foundation in the form of a Packard Research Fellowship.

\bibliographystyle{apj.bst}
\bibliography{references}


%\appendix
%\section{If needed}
%\label{app:warp}

\end{document}